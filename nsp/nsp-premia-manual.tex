%% -*- ispell-local-dictionary: "english"; -*-

%% This manual has been written by 
%% J�r�me Lelong 2008 INRIA

\documentclass[a4paper,11pt,twoside]{article}
\usepackage{a4wide}
\usepackage[dvips=true]{hyperref}
\usepackage{verbatim,moreverb,alltt}
\usepackage{makeidx}
\usepackage[dvips]{color}
\usepackage[latin1]{inputenc}
\usepackage[english]{babel}
\usepackage[strings]{underscore}



%% --------------------------------------------
\hbadness=10000
\emergencystretch=\hsize
\tolerance=9999
\parindent=0pt
%% --------------------------------------------

%\makeatletter
\long\def\varname#1#2{\item[#1]#2}
\def\var#1{{\tt #1}}

\newcommand{\describefun}[2]{%
  \index{#1}\phantomsection\item \label{#1} {\bf #1~(#2)}}
\newcommand{\describemeth}[2]{%
  \index{#1}\phantomsection\item \label{#1} {\bf #1[\if#2\empty ~\hskip1ex\else#2\fi]}}

\def\reffun#1{\hyperref[#1]{#1}}
\def\refmeth#1{\hyperref[#1]{#1[~]}}
\newenvironment{varlist}{\vskip1ex\noindent{\large\bfseries\color[named]{Blue} Parameters}%
  \unskip\begin{description}}{\end{description}\unskip}
\newenvironment{example}{\vskip1ex\noindent{\large\bfseries\color[named]{Blue} Examples}\unskip}{}
\def\code{\unskip\verbatim} \def\endcode{\endverbatim\unskip}
\newenvironment{describe}{\vskip1ex\noindent{\large\bfseries\color[named]{Blue} Description}%
  \vskip1ex\noindent}{} 
%\makeatother


\title{Nsp/Premia Manual}
\date{\today}
\author{Jean-Philippe Chancelier \and J\'er\^ome Lelong}

\makeindex
\begin{document}
\maketitle

\section{How Premia is embedded into Nsp}

For long, the only way to use Premia was from the command line. With the
growing of Premia every year, the need of real graphical user interface has
become more and more pressing. The idea of embedding the Premia library in a
Matlab-like Scientific Software has come up quite naturally. Unlike a
standalone graphical user interface, embedding Premia into Matlab-like
Scientific Software provides two ways of accessing the library either through
the scripting language or using the graphical capabilities of the
software. The possibility of accessing the Premia functions directly at the
interpreter level makes it possible to make Premia interact with other
toolboxes. Since the license of Premia gives right to freely distribute the
version of Premia two year older that the current release, it was important
that the scientific software used can be freely obtained and has extensive
graphical feature. Nsp fulfilled all these conditions.

The inheritance system of Nsp enables to easily add new objects in the
interpreter. This is how we introduced a new type named {\em PremiaModel},
through which the wide range of pricing problems described in Premia and their
corresponding pricing methods are made available from Nsp. The results
obtained in a given problem can be used in any post-treatment routines as any
other standard data.

For practitioners, the daily valuation of a complex portfolio is a burning
issue. Given a bunch of pricing problems to be solved, which are implemented
in Premia, how can we make the most of Nsp and the Premia toolbox? First, we
needed a way to describe a pricing problem in a way that is understandable by
Nsp so that it can create the correct instance of the {\em PremiaModel}
class. We implemented the {\em load} and {\em save} methods for such an
instance relying on the XDR library (eXternal Data Representation). This way,
any {\em PremiaModel} object can be saved to a file in a format which is
independent of the computer architecture; these files can be reloaded later by
any Nsp process. Then, a bunch of pricing problems can be represented by a
list of files created either from the scripting language or using the
graphical interface. Let us give an example of how to create such a file. To
save the pricing of an American call option in the one dimensional Heston
model using a finite difference method, one can use the following instructions
\begin{example}
  \begin{code}
    P = premia_create()
    P.set_asset[str="equity"]
    P.set_model[str="Heston1dim"]
    P.set_option[str="PutAmer"]
    P.set_method[str="FD_Fem_Achdou"]
    save('fic', P)
  \end{code}
\end{example}
Creating an instance of the {\em PremiaModel} class and setting its parameters
are very intuitive.  The object saved in the file \verb+fic+ can reloaded
using the command \verb+load('fic')+.


\section{A scripting approach}

The {\em PremiaModel} toolbox can be loaded into {\em Nsp} by running
\verb!exec loader.sce! in the directory \verb!nsp! of {\em Premia}.

\subsection{General functions}

In this section, argument written between square brackets are optional.

\begin{description}
  \describefun{premia_init}{}
  \begin{describe}
    It does some initialization stuff to create the {\em PremiaModel} type.
  \end{describe}

  \describefun{premia_create}{}
  \begin{describe}
    It creates an instance of a {\em PremiaModel} object and returns its address.
  \end{describe}
  
  \describefun{premia_get_assets}{}
  \begin{describe} It returns as a string matrix the names of the different
    asset type handled by Premia.
  \end{describe}
  
  \describefun{premia_get_models}{[asset=str]}
  \begin{varlist}
    \varname{str}{is a string parameter. It must be one of the values returned
      by \var{premia_get_assets}. This argument is optional and describes the
      type of underlying assets. When no \var{asset} argument is passed, the
      default value is \var{asset=equity}.}
  \end{varlist}
  \begin{describe} This function returns the list of the models available for
    the given asset type.
  \end{describe}

  \describefun{premia_get_families}{[n], [asset=str]}
  \begin{varlist}
    \varname{n}{is an integer parameter. This is the index of the family
      considered.}
    \varname{str}{is a string parameter. It must be one of the values
      returned by \var{premia_get_assets}.  This argument is optional and
      describes the type of underlying assets. When no \var{asset} argument
      is passed, the default value is \var{asset=equity}.}
  \end{varlist}
  \begin{describe} This function returns the name of the \var{n}$-th$ family
    available for the underlying asset type given by the optional and named
    argument \var{asset=str}. When the argument \var{n} is not given, the
    function returns the names of the different families available for the
    given asset type.
  \end{describe}

  \describefun{premia_get_family}{family, [model], [asset=str]}
  \begin{varlist}
    \varname{family}{is an integer parameter. This is the index of the family
      to be considered within the list returned by \reffun{premia_get_families}.}
    \varname{model}{is an integer parameter. This is the index of the model to
      be considered within the list returned by \reffun{premia_get_models}.}
    \varname{str}{is a string parameter. It must be one of the values returned
      by \var{premia_get_assets}.  This argument is optional and describes the
      type of underlying assets. When no \var{asset} argument is passed, the
      default value is \var{asset=equity}.}
  \end{varlist}
  \begin{describe} This function returns the list of options belonging the
    \var{family}$-th$ family in the \var{model}$-th$ model for the asset
    described by \var{str}. When no \var{model} argument is given, the names
    of all the options available in \var{family}$-th$ family are returned. If
    no option of the given family is compatble with the model choice, an empty
    string matrix is returned.
  \end{describe}
  
  \describefun{premia_get_methods}{family, option, model, [asset=str]}
  \begin{varlist}
    \varname{family}{is an integer describing the index of the family to be
      considered. This index can be guessed using
      \reffun{premia_get_families}.} 
    \varname{option}{is an integer describing the index of the option to be
      considered within the list of options available in the given
      family for the model defined by \var{model}.}
    \varname{model}{is an integer describing the index of the model to be
      considered. This index can be guessed using
      \reffun{premia_get_models}.} 
    \varname{str}{is a string parameter. It must be one of the values returned
      by \var{premia_get_assets}.  This argument is optional and describes the
      type of underlying assets. When no \var{asset}  argument is passed, the
      default value is \var{asset=equity}.}
  \end{varlist}
  \begin{describe} This function returns the names of the methods available to
    carry out the pricing problem defined by the arguments. The return value
    is a string matrix.
  \end{describe}

  \describemeth{load}{fic}
  \begin{varlist}
    \varname{fic}{is the name of a file}.
  \end{varlist}
  \begin{describe}
    It loads the {\em PremiaModel} object stored in the file \var{fic}. The
    file must have been created by the method \reffun{save}.
  \end{describe}
\end{description}

\subsection{PremiaModel's methods}

The following methods can be applied to a {\em PremiaModel} object \verb!M!
returned by the function \reffun{premia_create} using the syntax
\verb!M.method[]!  for instance. Note that unlike functions, parameters are
passed to methods within square brackets and not braces.

The symbol \textbar\ separates complementary parameters. This is actually a
concise way of describing two or possibly more ways of calling a method. The
use of curly braces to group parameters means that these parameters must
either be all used together or all ignored. Note that the curly braces must
not be typed into Nsp.

\begin{description}
  \describemeth{get_asset}{}
  \begin{describe}
    Returns the name of the underlying asset.
  \end{describe}
  
  \describemeth{get_model}{}
  \begin{describe}
    Returns the name of the model.
  \end{describe}

  \describemeth{get_models}{}
  \begin{describe}
    Returns the names of the models available for the already
    set asset type.
  \end{describe}

  
  \describemeth{get_option}{}
  \begin{describe}
    Returns the name of the option.
  \end{describe}

  \describemeth{get_family}{n}
  \begin{describe}
    Returns the names of the options of the $n-th$ family which can priced in
    the already set model. When no option is available in that family, an
    empty string matrix is returned.
  \end{describe}
  

  \describemeth{get_method}{}
  \begin{describe}
    Returns the name of the method.
  \end{describe}

  \describemeth{get_methods}{}
  \begin{describe}
    Returns the names of the methods available to solve the Premia problem
    described by the {\it PremiaModel} object.
  \end{describe}

  \describemeth{get_model_values}{}
  \begin{describe}
    Returns the list of parameters of the  model. 
  \end{describe}

  \describemeth{get_option_values}{}
  \begin{describe}
    Returns the list of parameters of the option.
  \end{describe}

  \describemeth{get_method_values}{}
  \begin{describe} Returns the list of parameters of the method. When the
    method does not need any parameter, an empty list is returned.
  \end{describe}

  \describemeth{set_asset}{str=asset}
  \begin{varlist}
    \varname{asset}{is a string parameter. It must be one of the values returned
      by \reffun{premia_get_assets}}
  \end{varlist}
  \begin{describe}
    It sets the underlying asset of a {\it PremiaModel} object.
  \end{describe}

  \describemeth{set_model}{n \textbar\ str=model}
  \begin{varlist}
    \varname{n}{is an integer parameter. It is the index of the
      model to be set in the list returned by \reffun{premia_get_models}
      called with the asset type of the object or \refmeth{get_models}.}
    \varname{model}{is a string parameter. It must be one of the values
      returned by \reffun{premia_get_models} called with the asset type of the
      object or \refmeth{get_models}.}
  \end{varlist}
  \begin{example}
    \begin{code}
      M=premia_create();
      M.set_asset[str="credit"];
      models=premia_get_models(asset=M.get_asset[])
      M.set_model[models(2)];
    \end{code}
    Equivalently, the list of available models for the already set asset type
    can be obtained by the method \refmeth{get_models}.
    \begin{code}
      M=premia_create();
      M.set_asset[str="credit"];
      models=M.get_models[];
      M.set_model[models(2)];
    \end{code}
  \end{example}
  \begin{describe}
    It sets the model of a {\it PremiaModel} object using either the parameter
    \var{n} or \var{str=model}.
  \end{describe}

  \describemeth{set_option}{$\{$n_family, n_option$\}$ \textbar\ str=option}
  \begin{varlist}
    \varname{n_family}{is an integer parameter. It is the index of the family
      in which the option is picked up. The index is computed within the
      restricted list of families compatible with the model already set.}
    \varname{n_option}{is an integer parameter. It is the index within the
      \var{n_family}$-th$ family of the option to be set. Note that the index
      is taken in the restricted list of options belonging to the given family
      which are compatible with the model}
    \varname{option}{is a string parameter. It must be one of the values
      returned by \reffun{premia_get_families} called with the parameters
      corresponding to the object.}
  \end{varlist}
  \begin{describe} It sets the option of a {\it PremiaModel} object using
    either the two parameters \var{n_option} and \var{n_family} or the unique
    parameter \var{str=option}. When the two parameter form is used, the
    option set is the \var{n_option}
  \end{describe}

  \describemeth{set_method}{n \textbar\ str=method}
  \begin{varlist}
    \varname{n}{is an integer parameter. This is the index of the method to be
      chosen in the restricted list of pricing methods available for the given
      asset, model ans option.}
    \varname{method}{is a string parameter. It must be one of the values
      returned by \reffun{premia_get_methods} called with the parameters
      corresponding to the object.}
  \end{varlist}
  \begin{describe} It sets the method of a {\it PremiaModel} object using
    either the parameter \var{n} or \var{str=method}. The list of availble
    methods can be withdrawn using the method \refmeth{get_methods}.
  \end{describe}
  
  \describemeth{set_model_values}{L}
  \begin{varlist}
    \varname{L}{is a list parameter. It must have the same structure as the one
      returned by \refmeth{get_model_values}}.
  \end{varlist}
  \begin{describe} It sets the different parameters of the model. The list
    \var{L} must define all the parameters. There is no way to specify just a
    few of them, instead you can withdraw the list returned by
    \refmeth{get_model_values}, modifiy a few values and set it back using
    \refmeth{set_model_values}.
  \end{describe}

  \describemeth{set_option_values}{L}
  \begin{varlist}
    \varname{L}{is a list parameter. It must have the same structure as the one
      returned by \refmeth{get_option_values}}.
  \end{varlist}
  \begin{describe} It sets the different parameters of the option. The list
    \var{L} must define all the parameters. There is no way to specify just a
    few of them, instead you can withdraw the list returned by
    \refmeth{get_option_values}, modifiy a few values and set it back using
    \refmeth{set_option_values}.
  \end{describe}

  \describemeth{set_method_values}{L}
  \begin{varlist}
    \varname{L}{is a list parameter. It must have the same structure as the one
      returned by \refmeth{get_method_values}}.
  \end{varlist}
  \begin{describe} It sets the different parameters of the method. The list
    \var{L} must define all the parameters. There is no way to specify just a
    few of them, instead you can withdraw the list returned by
    \refmeth{get_method_values}, modifiy a few values and set it back using
    \refmeth{set_method_values}.
  \end{describe}

  \describemeth{compute}{}
  \begin{describe} It computes the quantities of interest (price and delta
    generally) for the problem describe by the {\it PremiaModel} object. This
    is a void method. See \refmeth{get_method_results} to withdraw the results
    of the computation.
  \end{describe}

  \describemeth{get_compute_err}{}
  \begin{describe} If the computation run by \refmeth{compute} did not end
    successfuly, this method resturns the error message that occurred. When
    the computation ended sucessfully, an empty string matrix is returned.
  \end{describe}

  \describemeth{get_method_results}{}
  \begin{describe} It returns a list containing the values computed by the
    \refmeth{compute} method. If the \refmeth{compute} method has been called
    yest, an empty list is returned.
  \end{describe}

   \describemeth{get_method_results_iter}{}
   \begin{describe} It returns two arguments. The first one is a list of $2$
     or $3$ matrices depending on whether the iteration was performed on one
     ro two parameters. The first matrix (resp. first two matrices) contains
     the values taken by the parameter (the two parameters) on which we
     iterated. The last matrix contains the prices for the values of the
     parameter(s) given by the first (resp. first two) matrices. The second
     returned argument is a string matrix containing the names of the
     variables over which the iteration was performed.
   \end{describe}

   \describemeth{is_with_iter}{}
   \begin{describe}
     Returns a boolean : \verb!%t! if we have iterated on some parameters,
     \verb!%f! otherwise
   \end{describe}
   
  \describemeth{get_help}{type}
  \begin{varlist} \varname{type}{is a string parameter. It can be one of the
    keywords ``model'', ``option'' or ``method''.}
  \end{varlist}
  \begin{describe}
    It displays the PDF help corresponding to the object referred to by the
    \var{type} variable.
  \end{describe}

  \describemeth{save}{}
  \begin{describe}
    It saves a {\em PremiaModel} object to a file in a format that can be
    easily reloaded into Nsp (see function \reffun{load}). The format of the
    file is independent of the architecture.
  \end{describe}
\end{description}



\section{The graphical interface}

To access the graphical interface, one must execute the file
\verb!interface.sci! located in the directory \verb!nsp!.

\subsection{Basic usage}

\begin{description}
  \describefun{premia}{}
  \begin{describe}
    It launches the graphical interface and returns a {\em PremiaModel} object
    linked to the interface, which means that this object is modified eah time
    you change a parameter in the interface.
  \end{describe}
  
  \begin{example}
    \begin{code}
      P = premia ()
      // .. do some changes in the interface
      P.get_method[] //returns the name of the method
    \end{code}
  \end{example}
\end{description}

Once all the different parameters are properly set, the computation is
launched with the button  {\tt ``Compute''}.


\subsection{Advanced usage}

The interface has three menus : {\tt File, Help, Tools}, the use of which is
explained below.

\paragraph{File menu}

This menu provides the ability to save or reload a {\em PremiaModel} object
from a file. This is equivalent to using \reffun{load} or \reffun{save}.

\paragraph{Help menu}

This menu provides a fast and easy access to Premia's scientific
documentation. You can directly access the help concerning the model, the
option or the pricing method. This help is displayed in an external PDF viewer
as specified as compitation time (see the option \verb!--with-pdfviewer! of
the configure script).

\paragraph{Tools menu}

The graph menu is intended to be used with parameter iteration. It is possible
to use a vector for a parameter which is normally real valued and its means
that the prices must be computed for the different values specified in the
vector. The different results can be withdrawn using the method
\refmeth{get_method_results_iter} and are storred internally so that the
values can be plotted using the function {\tt Draw} from the {\tt Tools}
menu. If a second set of computations is carried out iterating on an other
parameter, the size of the two iterations must be the same because the new
results are concatenated to the previous ones and plotted on the same
graph. This is particularly useful to compare different methods for instance.

The function {\tt Clear Data} clears the variable in which the results of the
iterations are stored. The function {\tt Clear Graph} clears the graph which
can be redrawn using the function {\tt Redraw} as far as the {\tt Clear Data}
function has not been called.

Finally, it is possible to add a legend to the graph using the {\tt Legend}
function. The format of the legend is the folowing \verb!leg1@leg2@leg3...!
where \verb!leg1!, \verb!leg2! and $\dots$ are strings. The position of the
legend can also be specified 'ul' (upper left), 'ur' (upper right), ' dl'
(lower left) or 'dr' (lower right).



\printindex
\end{document}
