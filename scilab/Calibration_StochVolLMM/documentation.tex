\documentclass[12pt]{article}
%\linespread{1.6}

\usepackage{amsmath, amsthm, amsfonts, amssymb}
\setlength{\topmargin}{0cm} \setlength{\oddsidemargin}{0cm}
\setlength{\evensidemargin}{0cm} \setlength{\textwidth}{15truecm}
\setlength{\textheight}{22.8truecm}

\newtheorem{thm}{Theorem}[section]
\newtheorem{cor}[thm]{Corollary}
\newtheorem{lem}[thm]{Lemma}
\newtheorem{prop}[thm]{Proposition}
\newtheorem{example}[thm]{Example}
\newtheorem{remarks}[thm]{Remarks}
%\newtheorem{proof}[thm]{proof}

\theoremstyle{definition}
\newtheorem{defn}{Definition}[section]

\numberwithin{equation}{section} \theoremstyle{remark}
\newtheorem{rem}{Remark}[section]\def\vep{\varepsilon}
\def\<{\langle}
\def\>{\rangle}
\def\dsubset{\subset\subset}

\begin{document}

\title{Calibration of Libor market model with stochastic volatility: Implementation in PREMIA}
\vspace{4cm}
\author{ Xiao Wei\\[0.3cm]
Projet Mathfi, INRIA, 78153 Le Chesnay, France\\
School of Insurance, CUFE,  100081 Beijing, China}

\date{ }

\def\beq{\begin{equation}}
\def\nneq{\end{equation}}



\vspace{3cm}
\maketitle

\vspace{0.5cm}
\section{Introduction}
We consider an extension of Libor market model with a high-dimensional
Heston type stochastic volatility processes, which matches cap and swaption
volatility smiles and skews observed in the markets and allows for stable
calibration to the cap-strike matirx as well. Moreover the extension of
the model is made by preseving the local covariance structure of the
market model.\\

Here we use the road map proposed by
Belomestny, Mathew and Schoenmakers (2009) for calibration of this extension
Libor market model, which is FFT based, fast and easy to implement.\\

The content of this file is as follows. In section 2, we introduce the extended Libor
market model we considered in this paper and the pricing formula for capletbased on this
model is derived in section 3. Then we discribe the calibration road map to the caps and
pricing of swaptions in section 4. Section 5 is about the program manual of the
implementaion in PREMIA.\\


\section{Model Discription}
\subsection{The general Libor model}
For the tenor structure with a fixed sequence of tenor dates $0:=T_0<T_1<\cdots< T_n$, the Libor
forward rate is defined as:
$$
L_i(t)=\frac{1}{\delta_i}\left(\frac{B_i(t)}{B_{i+1}(t)}-1\right), \quad 0\leq t\leq T_i, 1\leq i<n
$$
where the day-count fractions $\delta_i:= T_{i+1}-T_i, i=1,2, \cdots, n-1,$ and
$B_i(t), i=1,2, \cdots, n, $ is the zero-conpon bond price at time $t$.\\

The dynamics of the Libor model is as follows:\\
\beq\label{lmm}
\frac{dL_i}{L_i}=(\cdots)dt+\gamma_i^{\top}dW, \quad i=1,2, \cdots, n-1,
\nneq
where the pre-speciafied volatility process $\gamma_i\in \mathbb{R}^m$ is adapted to the filtration
generated by the standard Brownian motion $W\in \mathbb{R}^m$, and the drift term, adumbrated by the dots,
is known under differenct numeraire measures, such as the risk-neutral, spot, terminal and all measures
induced by individual bonds taken as numeraire. We name the model (\ref{lmm}) as Libor {\it{market}} model,
when the volatility processes $t\to\gamma_i(t):=\gamma_i$ in (\ref{lmm}) are deterministic.\\

\subsection{The extended Libor market model}
The extension of Libor market model is proposed by adding a term of Heston type volatility on the extended Brownian motions
$\widetilde{W}$, then the dynamics of the extended Libor market model is given as
\begin{eqnarray}
& &\frac{dL_i}{L_i}=(\cdots)dt+\sqrt{1-r_i^2}\gamma_i^{\top}dW+r_i\beta_i^{\top}\sqrt{v}d\widetilde{W}, \quad 1\leq i< n\label{exlmm}\\
\label{vola}& & d\nu_k=\kappa_k (\theta_k-\nu_k)dt+\sigma_k\sqrt{\nu_k}\left(\rho_kd\widetilde{W}_k+\sqrt{1-\rho_k^2}d\overline{W}_k\right), \quad 1\leq k\leq d,
\end{eqnarray}
where $\widetilde{W}$ and $\overline{W}$ are mutually independent $d$-dimensional standard Brownian motions, both independent of $W$.
The
coefficients $r_i$ are constants that may be considered ``allotment" or "proportion" factor quantifying how much the original input market model should be in play. For $r_i=0$ for all $i$, the extended model restored to the standard Libor market model. As such, at nozero value of the $r_i$, the extended model may be regarded as a perturbation of the former. In this paper, we set for $i=1,2,\cdots, n,$ $r_i\equiv r$ and $\nu_k(0)\equiv\theta_k\equiv 1, 1\leq k< n$. \\

In model (\ref{exlmm}) the coefficients $\beta_i\in \mathbb{R}^d$ are choosen to be deterministic vector functions. To preseved the covariance of the Libor market model, it requires that
\beq\label{covreq}
\int_0^t\beta_i^{\top}\beta_jds=\int_0^t \gamma_i^{\top}\gamma_j ds,\quad 1\leq i, j<n.
\nneq
In order to obtain closed-form expressions for characteristic functions of (log-)Libors later on, we need $\beta(t)$ to be piecewise constant in time. But here for the progmatic sake, we choose of a constant vectors $\beta_i$ according to
\beq\label{exvola}
\beta_i=\sigma_i^{\mbox{Black}}e_i \quad \mbox{where}\quad \left(\sigma_i^{\mbox{Black}}\right)^2:=\frac{1}{T_i}\int_0^{T_i} |\gamma_i(s)|^2 ds,
\nneq
in order to match the covariance constraint (\ref{covreq}) roughly. Note that $\gamma_i\in\mathbb{R}^m$, even when $m<n-1$, matching (\ref{covreq}) may require $d=n-1$. In the calibration procedure, we assume the input market Libor volatility structure $\gamma\in \mathbb{R}^{(n-1)\times m}$ to be of full rank, that is $m=n-1$. So thoughout all our work we take $d=m=n-1$ and we also assume that $\beta\in \mathbb{R}^{(n-1)\times d}$ to be a squared upper triangular matrix.\\

\subsection{Extended model dynamics under terminal measure}
Donote the (time independent) solution of (\ref{exvola}) by $\overline{\gamma}\in \mathbb{R}^{(n-1)\times d}$. Let $r_i\equiv r$, under the terminal  measure $\mathbb{P}_n$, the dynamics of the extended Libor market model (\ref{exlmm}) is
\begin{eqnarray} \frac{dL_i}{L_i}=&-&\sum_{j=i+1}^{n-1}\frac{\delta_j L_j}{1+\delta_j L_j}\left[
(1-r^2)\gamma_i^{\top}\gamma_j+r^2\sum_{k=1}^{d} \overline{\gamma}_{ik}\overline{\gamma}_{jk}\nu_k
\right]dt\nonumber\\
&+&\label{modelterminal}\sqrt{1-r^2}\gamma_i^{\top}dW^{(n)}+r\sum_{k=1}^{d}\sqrt{\nu_k}\overline{\gamma}_{ik}d\widetilde{W}_k^{(n)}
\end{eqnarray} wtih the volativility as given by

\beq\label{terminalvola}
d\nu_k=\kappa_k (\theta_k-\nu_k)dt+\sigma_k\sqrt{\nu_k}\left(\rho_kd\widetilde{W}^{(n)}_k+\sqrt{1-\rho_k^2}d\overline{W}^{(n)}_k\right), \quad 1\leq k\leq d,
\nneq
\subsection{Extended model dynamics under forward measure}
For pricing caplet, we need to know the forward Libor dynamics in forward measure $P_{i+1}$. Then by rearraging terms in (\ref{modelterminal}) we have
\begin{eqnarray}
\frac{dL_i}{L_i}=&&\sqrt{1-r^2}\gamma_i^{\top}\left( dW^{(n)} -\sqrt{1-r^2}\sum_{j=i+1}^{n-1} \frac{1+\delta_j L_j}{\delta_jL_j}\gamma_j dt\right)\nonumber\\
&&+r\sum_{k=1}^d\overline{\gamma}_{ik}\sqrt{\nu_k}\left( d\widetilde{W}_k^{(n)} - \sum_{j=i+1}^{n-1} \frac{1+\delta_j L_j}{\delta_j L_j}\overline{\gamma}_{jk}\sqrt{\nu_k}dt \right)\nonumber\\
=:&&\sqrt{1-r^2}\overline{\gamma_i}^{\top}dW^{(i+1)} + r \sum_{k=1}^{d} \overline{\gamma_{ik}}\sqrt{\nu_k}d\widetilde{W}_k^{(i+1)}.\label{modelforward}
\end{eqnarray}\\

Since $L_i$ is a martingale under $P_{i+1}$, then both of $W^{(i+1)}$ and $\widetilde{W}^{(i+1)}$ in (\ref{modelforwad}) are standard Brownian motions under forward measure $P_{i+1}$. In term of these new Brownian motions the volatility dynamics are

\begin{eqnarray}
d\nu_k =&& \kappa_k(1-\nu_k)dt + r\sigma_k\rho_k \sum_{j=i+1}^{n-1} \frac{\delta_j L_j}{1+\delta_j L_j}\gamma_{jk}\nu_kdt\nonumber\\
&&+\rho_k\sigma_k\sqrt{\nu_k} d\widetilde{W}_k^{(i+1)} + \sqrt{1-\rho_k^2}\sigma_k\sqrt{\nu_k}d\overline{W}_k^{(n,i+1)},\label{volforward}
\end{eqnarray}
where $\overline{W}^{(n,i+1)}$ in (\ref{volforward}) is a standard Brownian motion under both measures $P_{i+1}$ and $P_n$. By freezing the Libors at their initial values in (\ref{volforward}), we obtain approximative CIR dynamic
\beq\label{volfreezing}
d\nu_k \approx \kappa_k^{i+1}\left( \theta_k^{(i+1)} -\nu_k \right)dt + \sigma_k\sqrt{\nu_k}\left( \rho_k d\widetilde{W}_k^{(i+1)} + \sqrt{1-\rho_k^2} d \overline{W}_k^{(i+1)}\right),\nneq
with reversion speed parameter
\beq\label{kappa}
\kappa_k^{(i+1)}:= \kappa_k -r \sigma_k\rho_k \sum_{j=i+1}^{n-1}\frac{\delta_jL_j(0)}{1+\delta_j L_j(0)}\overline{\gamma}_{jk},
\nneq
and mean reversion level
\beq\label{theta}
\theta_k^{(i+1)}:=\frac{\kappa_k}{\kappa_k^{(i+1)}}.
\nneq
We will use the above freezing volatility dynamics in the calibration routines.


\section{Pricing Caplet}

The arbitrage-free price at time zero $C_j(K)$ of a caplet at time $T_{j}, 1\leq j<n$ with strike $K$ paying $\delta_k(F_j(T_j)-X)^+$ at time $T_{j+1}$ is given by
\beq\label{price}C_j(K)=\delta_jB_{j+1}(0)E_{j+1}\left[(L_j(T_j)-K)^+\right],\nneq
where $E_{j+1}$ is the expectation taking under the forward measure $P_{i+1}$.\\

To calculate the caplet price (\ref{price}), we use the method of FFT-method of Carr and Madan (1999). The main result is as follows.\\

In term of the log-moneyness variable
\beq\label{logmoneyness} y:=\ln\left(\frac{K}{L_j(0)}\right),\nneq
the j-th caplet price can be expressed as
$$\mathbf{C}_j(y):=C_j(e^yL_j(0))=\delta_jB_{j+1}(0) L_j(0) E_{j+1}\left( e^{X_j(T_j) - e^y} \right)^+, $$
where $X_j(t) = \ln L_j(t) -\ln L_j(0)$ and we define the cahracteristic function of the process $X_j(t)$ under $P_{j+1}$ as $\varphi_{j+1}(\cdot; t)$.
Then for the auxiliary function
\beq\label{auxi} {\mathbf{O}}_j(y):= \delta_j^{-1}B_{j+1}^{-1}(0)L_j^{-1}(0)\it{C}_j(y)- (1-e^y)^+.\nneq
from Proposition 1 of Belomestny (2009), we have its Fourier trandform of the function $\mathbf{O}_j$ in term of $\varphi(\cdot; t)$ as
\beq\label{ft}
\mathbf{F}\{\mathbf{O}_j\}(z)= \int_{-\infty}^{\infty} \mathbf{O}_j(y)e^{iyz}dy=\frac{1-\varphi_{j+1}(z-i; T_j)}{z(z-i)}.
\nneq
Combining (\ref{logmoneyness}), (\ref{auxi}) and (\ref{ft}), we have the caplet price as
\begin{eqnarray}
C_j(K)=&&\delta_j B_{j+1}(0)(L_j(0)-K)^+\nonumber\\
&& +\frac{\delta_j B_{j+1}(0)L_j(0)}{2\pi}\int_{-\infty}^{\infty} \frac{1-\varphi_{j+1}(z-i; T_j)}{z(z-i)}e^{-iz \ln \left( \frac{K}{L_j(0)}\right)} dz.
\label{pricing}\end{eqnarray}
The explicit formula of the conditional characteristic funcion $\varphi_{j+1}(\cdot; t)$ is given asd in Appendix 8.0.1 of Belomestny et al (2009).

\section{Calibration algorithm}
The calibration algorithm is based on the pre-calibration Libor market model, which is calibrate to ATM caps and ATM swaptions using Schoenmakers (2005). The pre-calibration processure is not essential in this algorithm, we just use the pre-calibration result directly in our model. Here is pre-calibration volatility structure for Libor market model
$$\gamma_i(t)=c_ig_i(T_i -t) e_{i+1-m(t)},\quad \quad 0< t\leq T_i, 1\leq i\leq n,$$ where
function $$g(s)= g_{\infty} + (1-g_{\infty} +as) e^{-bs},$$ and $e_i\in \mathbb{R}^{n-1}$ are unit vectors such that $(e_{i,k})$ is upper triangular.
For the pre-calibrated Libor market model the loading factors $c_i$ are readily computed from
\beq\label{lf}
\left(\sigma_{T_i}^{ATM}\right)^2 T_i = c_i\int_0^{T_i} g^2(s)ds, \quad \quad i=1,\cdots, n-1.
\nneq
In this calibration routine, the loading factors $c_i$ will be calibrated as newly for flexibility.
We will use the pre-calibration result $g_{\infty}=2.578, a= 5.001, b=2.000$ for the Libor market model in the our calibration algorithm.\\


We will calibrate the model parameters sets $\{\kappa_i, \sigma_i, \rho_i, c_i; 1\leq i <n \}$ and $r$ to the market cap-strike volatility matrix by an iteration step, where $\kappa_i, \sigma_i, \rho_i, 1\leq i <n$ are parameters in the volatility dynamics (\ref{volforward}) and $c_i, 1\leq i <n$ are the loading factors in Libor market model. At each step we minimizing the weight sum of the difference between the Black-scholes caplet price and the computed caplet price using (\ref{pricing}) for different strike, the weight are taken to be proportional to Black-Scholes vegas.\\

The iteration step is as follows:\\

\begin{itemize} \item We star from step $i=n-1$. Calibrate $r$ and the parameter set $(\kappa_{n-1}, \sigma_{n-1}, \rho_{n-1}, c_{n-1})$ to the $T_{n-1}$ column of the cap-strike matirx using teh explicitly known characteristic function $\varphi_n$ of the $\ln[L_{n-1}(T_{n-1}/L_{n-1}(0))]$.\\
\item For the steps from $i=n-2$ down to $1$ we carry out the next iteration step:\\
\item For the $k$-th step $i=n-k$, transform the yet know parameter set $(\kappa_j, \sigma_j, \rho_j, c_j), i<j<n$  via (\ref{kappa}) and (\ref{theta}) into the corresponding set $(\kappa_j^{(i+1)}, \sigma_j^{(i+1)}, \rho_j^{(i+1)}, c_j), i<j<n$. By teh upper triangular structure of the square matrix $\overline{\gamma}$ we obviously have $\kappa_i^{(i+1)}=\kappa_i$ and hense by (\ref{theta}) $\theta_i^{(i+1)}=1$. Then calibrate the unknown parameter set $(\kappa_j, \sigma_j, \rho_j, c_j), i<j<n$ to the $T_i$ column of the cap-strike matrix using (\ref{pricing}).\\
\end{itemize}


\section{Program manual}
\noindent {\bf{Including files:}}\\
\noindent The program directory contains 22 files:\\
\begin{itemize}
\item this documentation file.
\item the source program file: ``calibrate$\_$lmm$\_$vola$\_$stoc.cpp" written in C++
\item a binary executable file ``calibrate.out" generated by\\
the compilation of ``calibrate$\_$lmm$\_$vola$\_$stoc.cpp"
\item an output file of calibration results: ``cali$\_$result"
\item an example of input market data: ``initial$\_$curve.dat"
``cap$\_$strike$\_$matrix.dat"
\item header and link files (altogether 16 files):``intg.h"   ``intg.C"   ``normal$\_$df.h"
``normal$\_$df.C" ``error$\_$msg.h" ``optype.h"
 ``realfft.h"    ``realfft.C"  ``min.h"   ``min.cpp"   ``routines.c"
``routines.cpp"   ``routines.o"   ``f2c.h"   ``libf2c.a"
``iterate"
\end{itemize}

\noindent {\bf{Compilation:}}\\
\noindent Compilation command under Linux: g++ intg.C normal$\_$df.C
realfft.C min.cpp routines.cpp libf2c.a
cali$\_$lmm$\_$vola$\_$stoc.cpp -o calibrate.out, generates the
executable binary file
``calibrate.out".\\


 \noindent{\bf{Execution:}}\\
\noindent Type ./calibrate.out to run the calibration program.\\

 Five parameters can be calibrated to each line of the cap-strike volatility matrix in this program:
\begin{itemize}
\item r: $0\leq r \leq 1$

\item ke: $0 \leq ke \leq 2$

\item sigma: $0.0006 \leq sigma \leq 2$

\item rho: $-0.9999 \leq rho \leq 0.9999$

\item lf:$ 0.1 \leq lf \leq 4$
\end{itemize}

\noindent Results are pressented at the end of program and in the
result file "cali$\_$result"

 \end{document}
