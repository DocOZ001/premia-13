\documentclass[12pt]{article}

\usepackage{a4}
\usepackage{amsmath}
\usepackage{english}
\usepackage{rotating}
\usepackage{epsfig}
\usepackage{graphicx}
\usepackage{float}
\usepackage{amssymb}
\usepackage{array}

\newcommand{\nc}{\newcommand}
\nc{\rnc}{\renewcommand}

\nc{\draft}{true}
\nc{\draftfig}{false}

\nc{\dsp}{\displaystyle}

\newtheorem{Rem}{Remarque}
\newtheorem{lemme}{Lemme}
\newtheorem{defi}{D\'efinition}
\newtheorem{theo}{Th\'eor\`eme}
\newtheorem{corr}{Corrolaire}

\newcommand {\bsigma}{\bar{\sigma}}
\newcommand {\vertBox}[1]{\begin{sideways}\fbox{#1}\end{sideways}}

\newcommand{\noi}{\noindent}
\newcommand{\bs}{\bigskip}
\newcommand{\ms}{\medskip}
\newcommand{\debdem}{{\bf D\'emonstration.}\quad}
\newcommand\findem{\hfill{$\blacksquare$}\medskip}

% \newcommand{\debdem}{\begin{proof}} \def\findem{\end{proof}}

\newcommand{\refeq}[1]{(\ref{#1})}
%\newcommand{\eqref}[1]{(\ref{#1})}

\def\sharpo{{ }}

\def\1B{{\bf  1}}
\def\argmin{\mathop{\rm argmin}}
\def\argmax{\mathop{\rm argmax}}
\def\dw{\mathop{\delta w}}

\def\cl{\mathop{\rm cl}}
\def\conv{\mathop{\rm conv}}
\def\diag{\mathop{\rm diag}}
\def\dist{\mathop{\rm dist}}
\def\dom{\mathop{{\rm dom}}}
\def\supp{\mathop{\rm supp}}
\def\val{\mathop{\rm val}}
\def\var{\mathop{\rm var}}
\def\epi{\mathop{\text{\'epi}}}
\def\isom{\mathop{\rm Isom}}
\def\inv{\mathop{\rm inv}}
\def\intt{\mathop{\rm int}}
\def\Ker{\mathop{\rm Ker}}

\def\Inf{\mathop{\rm Inf}}
\def\Min{\mathop{\rm Min}}
\def\Max{\mathop{\rm Max}}


\def\rar{\rightarrow}
\def\trace{\mbox{trace}}

\newcommand\be{\begin{equation}}
\newcommand\ee{\end{equation}}
\newcommand\ba{\begin{array}}
\newcommand\ea{\end{array}}
\def\disp{\displaystyle}
% \def\cR{I\hspace{-1mm}R}
\def\cZ{\mathbb{Z}}

\newcommand{\cE}{I\!\!E}
\newcommand{\cR}{I\!\! R}
\newcommand{\cRbar}{\bar {I\!\! R}}
\newcommand{\cN}{I\!\!N}

\def\Om{{\Omega}}
\def\om{{\omega}}
\def\ml{\lambda}
\def\mlb{\bar{\lambda}}
\def\mub{\bar{\mu}}
\def\nub{\bar{\nu}}

\def\Cbar{\bar{C}}
\def\Tbar{\bar{T}}

\def\cL{{\cal L}}
\def\cP{{\cal P}}
\def\cPt{\widetilde{\cal P}}

\def\ph{\hat{p}}
\def\qh{\hat{q}}
\def\rh{\hat{r}}
\def\sh{\hat{s}}
\def\th{\hat{t}}
\def\uh{\hat{u}}
\def\vh{\hat{v}}
\def\wh{\hat{w}}
\def\xh{\hat{x}}
\def\yh{\hat{y}}

\def\db{\bar{d}}
\def\hb{\bar{h}}
\def\pb{\bar{p}}
\def\qb{\bar{q}}
\def\rb{\bar{r}}
\def\sb{\bar{s}}
\def\tb{\bar{t}}
\def\ub{\bar{u}}
\def\vb{\bar{v}}
\def\wb{\bar{w}}
\def\xb{\bar{x}}
\def\yb{\bar{y}}

\def\ft{\tilde{f}}
\def\ut{\tilde{u}}
\def\xt{\tilde{x}}

\def\eps{\varepsilon}
\def\La{\langle}
\def\la{\langle}
\def\ra{\rangle}
\def\half{\mbox{$\frac{1}{2}$}}
\def\sixinv{\mbox{$\frac{1}{6}$}}

\def\A{\mathcal{A}}
\def\C{\mathcal{C}}
\def\D{\mathcal{D}}
\def\F{\mathcal{F}}
\def\G{\mathcal{G}}
\def\H{\mathcal{H}}
\def\K{\mathcal{K}}
\def\L{\mathcal{L}}
\def\M{\mathcal{M}}
\def\N{\mathcal{N}}
\def\PP{\mathcal{P}}
\def\P{\mathcal{P}}
\def\Q{\mathcal{Q}}
\def\R{\mathcal{R}}
\def\SS{\mathcal{S}}
\def\T{\mathcal{T}}
\def\U{\mathcal{U}}
\def\V{\mathcal{V}}

\def\tV{\tilde{V}}

\newenvironment{myenumerate}{
\renewcommand{\theenumi}{\roman{enumi}}
\renewcommand{\labelenumi}{(\theenumi)}
\begin{enumerate}}{\end{enumerate}}

% EOF


\title{Documentation of the calibration's code}
\author{Sophie Volle and Jean-Marc Cognet \\ \\
INRIA Rocquencourt - MATHFI project \\ 
Domaine de Voluceau \\ 
B.P. 105 \\
78153 Le Chesnay Cedex}
\date{november 2002}

\begin{document}

\maketitle

\tableofcontents
\clearpage

\section{Introduction}

The algorithm of calibration described in~\cite{bcv:inria:02} has 
been implemented in C. The five main programs contained in the 
subdirectory ./src are:
\begin{itemize}
\item 
simul.c: (pricing) computes option prices with Dupire's PDE or 
Black-Scholes formulae. This program is used to compare 
Dupire/Black-Scholes and to build synthetic data;
\item
calib.c: (calibration) computes a local volatility from option 
prices;
\item
rafsig.c: computes the degrees of freedom of the volatility on a finer 
grid. This program is used to perform a multiscale approach;
\item
visusig.c: allows us to visualize a volatility from its bicubic 
spline degrees of freedom;
\item
impsig.c: computes implied volatilities.
\end{itemize}
Each program is described with more details in 
section~\ref{SEC:PROGRAMS}. Notice that each program is controlled 
by a file ended by .in (e.g. simul.in for the program simul.c) 
containing the parameters and the names of the in/out files.

\section{Compilation, execution and useful scripts}

The source files of the calibration code are in the 
subdirectory: ./src. To compile the five programs simul, calib, 
rafsig, visusig and impsig:

\begin{verbatim}
cd ./src
make or make all
\end{verbatim}
or you can compile each program separately:
\begin{verbatim}
make simul
make calib
etc.
\end{verbatim}

You can execute each program from other test subdirectories 
(e.g. by creating ./tests). Make sure that the subdirectory 
./src containing the programs (after compilation) is in 
your Unix environment variable PATH. 

To execute a program, e.g. calib, from a test subdirectory, 
you have to edit a file ended by .in, calib.in in this case. 
You can find examples of files simul.in, calib.in, etc. 
in the subdirectory ./src/examples.

The subdirectory ./src/matlab contains useful programs 
to visualize the results with Matlab. For instance, 
vprice.m (type help vprice in Matlab to have the documentation) 
allows us the visualize option prices or volatilities 
(files ended by .visu in the given examples).

The Perl scripts decroi1 and decroi2 contained in ./src/scripts 
are useful to extract the decrease of the cost function from 
a file containing the output of a calib execution. Use decroi1 
if the output is given by the Quasi-Newton algorithm without 
constraints and decroi2 if the output is given by the Quasi-Newton 
algorithm with bounds.

The subdirectory ./doc contains this documentation. \\
See the file ./doc/documentation\_code.ps. 

\section{Implementation}

We give here the list of the files containing the different routines
used for:
\begin{itemize}
\item
Dupire's PDE and bicubic splines: DupirePDE.c, solveSystem.c, 
sparse.c, spline.c;
\item
Objective function and its gradient: objFunction.c, grad.c, 
gradFiniteDiff.c, utilsGrad.c. Notice that, in this version of 
the code, the gradient of the objective function is computed by 
finite differences (gradFiniteDiff.c). However, a computation 
using an adjoint state technique has been implemented and tested 
(grad.c, utilsGrad.c) but the code, which has not been 
optimized, is lower than a computation by finite differences. 
The computation by adjoint state could be improved in a 
next version;
\item
Quasi-Newton without constraints: QuasiNewton.c, BFGSupdate.c, 
cholesky.c, lineSearch.c, stopping.c, testing.c. 
See Barette~\cite{bar:optim:02} for details concerning this 
optimization algorithm;
\item
Quasi-Newton with bounds: routines.f. See Nocedal 
{\em et~al.}~\cite{noce:siam:95,noce:acm:97} for more details. 
Notice that this code written in Fortran has been interfaced with 
our C program calib.c;
\item 
in/out files: inout.c. 
\end{itemize}

\section{Programs}
\label{SEC:PROGRAMS}

\subsection{Program simul}
\label{SSEC:SIMUL}

The program simul computes option prices from given $S_0$, $r$, $q$ 
and volatility ($\sigma$) parameters :
\begin{itemize}
\item
using Dupire's PDE; 
\item
using Black-Scholes formulae with $\sigma = {\rm Constant}$;
\item
using Black-Scholes formulae with $\sigma = \sigma(t)$.
\end{itemize}
If Dupire is used, $\sigma$ can be given by the function sigma\_func 
(defined in spline.c) or by its bicubic spline degrees of freedom 
on a coarse grid $n \times m$, given by the file name\_ddl. This 
file contains: 
$n$, $m$, $y_0$, ..., $y_n$, $t_0$, ..., $t_m$, ddl $\rightarrow$ 
$(n+3)(m+3)$ values.\\
If Black-Scholes (with $\sigma = {\rm Constant}$) is used, $\sigma$ is 
given by sigmaCte.\\
If Black-Scholes (with $\sigma = \sigma(t)$) is used, the function 
intSigma (defined in DupirePDE.c) has to be defined:
$$
{\rm intSigma} = \frac{1}{T-t_0}\int_{t_0}^T \sigma^2(t)dt.
$$
The option prices can be computed:
\begin{itemize}
\item
on a rectangular grid given by the file name\_in\_visu. This file 
contains:
$n_{visu}$, $m_{visu}$, $K_0$, ..., $K_{n_{visu}}$, $t_0$, ..., 
$t_{m_{visu}}$. In this case, the prices are saved in the file 
name\_out\_visu: $n_{visu}$, $m_{visu}$, $K_0$, ..., $K_{n_{visu}}$, 
$t_0$, ..., $t_{m_{visu}}$, $V_{0,0}$, ..., $V_{0,m_{visu}}$, ..., 
$V_{n_{visu},0}$, ..., $V_{n_{visu},m_{visu}}$  
where 
$$
V_{i,j} = V(K_i,T_j);
$$
\item
for strikes-maturities $(K_i,T_j)$ given by the file name\_in\_data.
In this case, the prices are saved in the file name\_out\_data containing 
three columns: $K_i$ $T_j$ $V_{i,j}$.
\end{itemize}

The input file simul.in contains the following data:
\begin{verbatim}
###################### VARIABLES S_0, R, Q AND OPTIONTYPE ######################
S_0 : price of the underlying asset at t=t_0
r : risk-free rate     
q : dividends (continuously compounded) 
optionType : type of the option (1 for call, 0 for put)  
########################### DUPIRE OR BLACK-SCHOLES ? ##########################
optionSimul :
  1 for PDE Dupire                --> t_0, T_max, y_min, y_max, N, M, gridType
                                  --> theta, name_ddl or sigma_func (spline.c)
  2 for Black-Scholes (sigma cte) --> sigmaCte
  3 for Black-Scholes (sigma(t))  --> intSigma (DupirePDE.c)
########################### IF OPTIONSIMUL=1 (DUPIRE) ##########################
t_0 : time origin
T_max : maximal maturity (in year)
y_min : --> S_min=exp(y_min)     
y_max : --> S_max=exp(y_max)
N : number of space steps of the fine grid
M : number of time steps of the fine grid
gridType : type of the y_fineGrid (0 for regular, 1 for tanh)
theta : type of scheme used
name_ddl : name of the file containing the degrees of freedom of sigma
  Type "_" if sigma has to be defined from the function sigma_func (spline.c)
############# IF OPTIONSIMUL=2 (B-S WITH SIGMA CONSTANT) #######################
sigmaCte : value of sigma
################################# FILES IN/OUT #################################
name_in_visu : name of the file containing the visualization parameters
  Type "_" if no visualization is required
name_out_visu : name of the out file containing the prices to be visualized
  used if name_in_visu is not given by "_"
name_in_data : name of the file containing Ki and Tj
  Type "_" if no data have to be computed
name_out_data : name of the out file containing Ki, Tj and the prices Vij
  used if name_in_data is not given by "_"
\end{verbatim}

\subsection{Program calib}
\label{SSEC:CALIB}

The program calib computes the volatility $\sigma$ from given values of 
$S_0$, $r$, $q$ and option prices, by minimizing the following cost 
function: 
$$
F(\sigma) = G(\sigma) + \lambda F_1(\sigma) 
= \frac{1}{2} \| V(\sigma) - \tilde{V} \|^2 + 
\lambda \| \nabla \sigma \|^2
$$
with : 
\begin{itemize}
\item
$V(\sigma)$ denotes the computed option prices with Dupire's PDE;
\item
$\tilde{V}$ the option prices given by the market;
\item
$\|.\|$ the Euclidean norm;
\item
$\lambda$ the coefficient of regularization.
\end{itemize}
The data prices are contained in the file name\_in\_data which has three 
columns: $K_i$ $T_j$ $\tilde{V}_{i,j}$ with 
$$
\tilde{V}_{i,j} = \tilde{V}(K_i,T_j).
$$
The minimization algorithm is defined by choice\_optim:
\begin{itemize}
\item
$1$ $\rightarrow$ Quasi-Newton without constraints (implemented by 
V. Barette~\cite{bar:optim:02});
\item
$2$ $\rightarrow$ Quasi-Newton with bounds (implemented by Nocedal 
et al.~\cite{noce:siam:95,noce:acm:97}).
\end{itemize}
In each case, some parameters concerning the optimization can be 
defined by the file name\_in\_optim.\\
If ${\rm choice\_optim} = 1$ (Q-N), name\_in\_optim contains:
\begin{itemize}
\item 
gradtol : tolerance on the relative gradient;
\item 
steptol : tolerance on the relative change of $x$;
\item 
verbosity : level of printed information ($0 \rightarrow 3$);
\item 
saveSuccessiveXinFile : save successive x0 in the file data.out 
($0$ or $1$);
\item 
maxCounter : maximum number of iterations;
\item 
$\lambda$ : coefficient of $F_1$ in the objective function: 
$F = G + \lambda F_1$.
\end{itemize}
If ${\rm choice\_optim} = 2$ (Q-N with bounds), name\_in\_optim 
contains:
\begin{itemize}
\item 
pgtol : tolerance on the infinity norm of the projected gradient;
\item 
factr : tolerance on the change of the objective function;
\item 
iprint : level of printed information;
\item 
maxCounter : maximum number of iterations;
\item 
$\sigma_{min}$ : minimal bound on sigma;
\item 
$\sigma_{max}$ : maximal bound on sigma;
\item
$\lambda$ : coefficient of $F_1$ in the objective function: 
$F = G + \lambda F_1$.
\end{itemize}
The initial volatility is given by its bicubic spline degrees of 
freedom on a coarse grid $n \times m$, given by the file 
name\_in\_sigmainit\_ddl. This file contains: 
$n$, $m$, $y_0$, ..., $y_n$, $t_0$, ..., $t_m$, ddl $\rightarrow$ 
$(n+3)(m+3)$ values. At the end of the calibration algorithm, the 
degrees of freedom are saved in the file name\_out\_sigmaest\_ddl. 
In order to visualize the volatilities, the initial and the estimated 
$\sigma$ can also be computed on a rectangular grid given by the file 
name\_in\_sigma\_visu. This file contains: 
$n_{visu}$, $m_{visu}$, $S_0$, ..., $S_{n_{visu}}$, $t_0$, ..., 
$t_{m_{visu}}$. The values of the initial and the estimated $\sigma$ on 
the rectangular grid are saved in the files name\_out\_sigmainit\_visu 
and name\_out\_sigmaest\_visu respectively. These files contain:
$n_{visu}$, $m_{visu}$, $S_0$, ..., $S_{n_{visu}}$, $t_0$, ..., 
$t_{m_{visu}}$, $\sigma_{0,0}$..$\sigma_{0,m_{visu}}$ ... 
$\sigma_{n_{visu},0}$..$\sigma_{n_{visu},m_{visu}}$ where 
$$
\sigma_{i,j} = \sigma_{S_i,t_j}.
$$

The input file calib.in contains the following data:
\begin{verbatim}
###################### VARIABLES S_0, R, Q AND OPTIONTYPE ######################
S_0 : price of the underlying asset at t=t_0
r : risk-free rate     
q : dividends (continuously compounded) 
optionType : type of the option (1 for call, 0 for put)  
##################################### DUPIRE ###################################
t_0 : time origin
T_max : maximal maturity (in year)
y_min : --> S_min=exp(y_min)     
y_max : --> S_max=exp(y_max)
N : number of space steps of the fine grid
M : number of time steps of the fine grid
gridType : type of the y_fineGrid (0 for regular, 1 for tanh)
theta : type of scheme used
################################## OPTIMIZATION ################################
choice_optim : choice of the optimization method (1-->QN, 2-->QN with bounds)
################################# FILES IN/OUT #################################
name_in_optim : name of the file containing the optimization parameters
name_in_data : name of the file containing the data prices
name_in_sigmainit_ddl : name of the file containing the degrees of freedom of
  the initial sigma
name_out_sigmaest_ddl : name of the file containing the degrees of freedom of
  the estimated sigma
name_in_sigma_visu : name of the file containing the visualization parameters
  Type "_" if no visualization is required
name_out_sigmainit_visu : name of the out file containing the initial sigma 
  to be visualized (Type "_" is no visualization is required)
name_out_sigmaest_visu : name of the out file containing the estimated sigma
  to be visualized (Type "_" is no visualization is required)
\end{verbatim}

\subsection{Program rafsig}
\label{SSEC:RAFSIG}

The program rafsig computes the bicubic spline degrees of freedom 
of the volatility $\sigma$ on a grid $n_2 \times m_2$, from its degrees 
of freedom on a coarser grid $n_1 \times m_1$. The volatility on the grid 
$n_1 \times m_1$ is given by the file name\_in\_sigma1\_ddl. This file 
contains: $n_1$, $m_1$, $y_0$, ..., $y_{n_1}$, $t_0$, ..., $t_{m_1}$, 
ddl $\rightarrow$ $(n+3)(m+3)$ values. The grid $n_2 \times m_2$ is 
obtained by splitting the grid $n_1 \times m_1$ in $nbsplit_y$ 
subdivisions in $y$ and $nbsplit_T$ subdivisions in $T$. The subdivisions 
are regular on the grid $[K,T]$: 
Grid1 $[Y,T]$ $\rightarrow$ Grid1 $[K,T]$ splitted regularly w.r.t 
$nbsplit_y$ and $nbsplit_T$ $\rightarrow$ Grid2 $[K,T]$ $\rightarrow$ 
Grid2 $[Y,T]$. The volatility on the grid $n_2 \times m_2$ is saved in the 
file name\_out\_sigma2\_ddl: 
$n_2$, $m_2$, $y_0$, ..., $y_{n_2}$, $t_0$, ..., $t_{m_2}$, 
ddl $\rightarrow$ $(n+3)(m+3)$ values.

The input file rafsig.in contains the following data:
\begin{verbatim}
################################# FILES IN/OUT #################################
name_in_sigma1_ddl : name of the file containing the ddl
nbsplit_y : number (integer) of subdivisions in y
nbsplit_T : number (integer) of subdivisions in T
name_out_sigma2_ddl : name of the out file containing the new ddl
\end{verbatim}

\subsection{Program visusig}
\label{SSEC:VISUSIG}

The program visusig computes the volatility $\sigma$ on a rectangular 
grid from its bicubic spline degrees of freedom. It is useful to 
visualize $\sigma$. The volatility on the grid $n \times m$ is given by 
the file name\_in\_sigma\_ddl. This file contains: 
$n$, $m$, $y_0$, ..., $y_n$, $t_0$, ..., $t_m$, 
ddl $\rightarrow$ $(n+3)(m+3)$ values. The rectangular grid is given by 
the file name\_in\_sigma\_visu containing: 
$n_{visu}$, $m_{visu}$, $S_0$, ..., $S_{n_{visu}}$, $t_0$, ..., 
$t_{m_{visu}}$. The values of $\sigma$ on the rectangular grid are saved 
in the file name\_out\_sigma\_visu: 
$n_{visu}$, $m_{visu}$, $S_0$, ..., $S_{n_{visu}}$, $t_0$, ..., 
$t_{m_{visu}}$, $\sigma_{0,0}$..$\sigma_{0,m_{visu}}$ ... 
$\sigma_{n_{visu},0}$..$\sigma_{n_{visu},m_{visu}}$ where 
$$
\sigma_{i,j} = \sigma_{S_i,t_j}.
$$

The input file visusig.in contains the following data:
\begin{verbatim}
################################# FILES IN/OUT #################################
name_in_sigma_ddl : name of the file containing the ddl
name_in_sigma_visu : name of the file containing the visualization parameters
name_out_sigma_visu : name of the out file containing sigma 
\end{verbatim}

\subsection{Program impsig}
\label{SSEC:IMPSIG}

The program impsig computes the implied volatilities from given 
$S_0$, $r$, $q$, optionType, $t_0$ and option prices. The data prices 
are contained in the file name\_in\_data which has three 
columns: $K_i$ $T_j$ $V_{i,j}$ with 
$$
V_{i,j} = V(K_i,T_j).
$$
The implied volatilities are saved in the file name\_out\_sigma which 
has three columns: $K_i$ $T_j$ $\sigma_{i,j}$ with 
$$
\sigma_{i,j} = {\rm Implied\_Sigma}(K_i,T_j).
$$

The input file impsig.in contains the following data:
\begin{verbatim}
#################### VARIABLES S_0, R, Q, OPTIONTYPE AND T_0 ###################
S_0 : price of the underlying asset at t=t_0
r : risk-free rate     
q : dividends (continuously compounded) 
optionType : type of the option (1 for call, 0 for put)  
t_0 : time origin
################################# FILES IN/OUT #################################
name_in_data : name of the file containing Ki, Tj and V_ij
name_out_sigma : name of the file containing Ki, Tj and impsigma_ij
\end{verbatim}

\section{Examples}

The following files are contained in the subdirectory ./src/example.

\subsection{simul.in}

This is the file simul.in used to build the synthetic data with 
$$
\sigma_{true}(S,t) = 0.05 
+ 0.1 \exp \left( -\frac{S}{100} \right) + 0.5 t
$$
(see~\cite{bcv:inria:02}).

\begin{verbatim}
################################################################################
###################### VARIABLES S_0, R, Q AND OPTIONTYPE ######################
################################################################################
# S_0 : price of the underlying asset at t=t_0
100
# r : risk-free rate     
0.05
# q : dividends (continuously compounded) 
0.02
# optionType : type of the option (1 for call, 0 for put)  
1
################################################################################
########################### DUPIRE OR BLACK-SCHOLES ? ##########################
################################################################################
# optionSimul :
#   1 for PDE Dupire                --> t_0, T_max, y_min, y_max, N, M, gridType
#                                   --> theta, name_ddl or sigma_func (spline.c)
#   2 for Black-Scholes (sigma cte) --> sigmaCte
#   3 for Black-Scholes (sigma(t))  --> intSigma (DupirePDE.c)
1
################################################################################
########################### IF OPTIONSIMUL=1 (DUPIRE) ##########################
################################################################################
# t_0 : time origin
0
# T_max : maximal maturity (in year)
1
# y_min : --> S_min=exp(y_min)     
-5.29831736654804
# y_max : --> S_max=exp(y_max)
5.29831736654804
# N : number of space steps of the fine grid
400
# M : number of time steps of the fine grid
100
# gridType : type of the y_fineGrid (0 for regular, 1 for tanh)
0
# theta : type of scheme used
.5
# name_ddl : name of the file containing the degrees of freedom of sigma
#   Type "_" if sigma has to be defined from the function sigma_func (spline.c)
_
################################################################################
############# IF OPTIONSIMUL=2 (B-S WITH SIGMA CONSTANT) #######################
################################################################################
# sigmaCte : value of sigma
.2
################################################################################
################################# FILES IN/OUT #################################
################################################################################
# name_in_visu : name of the file containing the visualization parameters
#   Type "_" if no visualization is required
visuprices
# name_out_visu : name of the out file containing the prices to be visualized
#   used if name_in_visu is not given by "_"
dupireprices.visu
# name_in_data : name of the file containing Ki and Tj
#   Type "_" if no data have to be computed
dataprices
# name_out_data : name of the out file containing Ki, Tj and the prices Vij
#   used if name_in_data is not given by "_"
dupireprices.data
\end{verbatim}

The files visuprices and dataprices are also contained in the subdirectory 
./src/examples. The function sigma\_func defined in spline.c is:
\begin{verbatim}
double sigma_func(double S, double t){
/* OUTPUT
   - returns sigma(S,t)
   INPUTS
   - S
   - t   */
  return 0.01+0.1*exp(-S/100)+0.01*t;
}
\end{verbatim}
After execution of simul, the output files are dupireprices.visu and 
dupireprices.data. The latter file is:
\begin{verbatim}
90.000000 0.500000 13.149445
92.000000 0.500000 11.726255
94.000000 0.500000 10.392428
96.000000 0.500000 9.149493
98.000000 0.500000 7.997157
100.000000 0.500000 6.956789
102.000000 0.500000 6.018150
104.000000 0.500000 5.166235
106.000000 0.500000 4.395305
108.000000 0.500000 3.737694
110.000000 0.500000 3.154494
90.000000 1.000000 20.477956
92.000000 1.000000 19.401076
94.000000 1.000000 18.367679
96.000000 1.000000 17.376526
98.000000 1.000000 16.426167
100.000000 1.000000 15.521834
102.000000 1.000000 14.659627
104.000000 1.000000 13.834786
106.000000 1.000000 13.045292
108.000000 1.000000 12.301597
110.000000 1.000000 11.592354
\end{verbatim}

\subsection{calib.in}

This is the file calib.in used to estimate a local volatility from 
real option prices (see~\cite{bcv:inria:02}).

\begin{verbatim}
################################################################################
###################### VARIABLES S_0, R, Q AND OPTIONTYPE ######################
################################################################################
# S_0 : price of the underlying asset at t=t_0
590
# r : risk-free rate     
0.06
# q : dividends (continuously compounded) 
0.0262
# optionType : type of the option (1 for call, 0 for put)  
1
################################################################################
##################################### DUPIRE ###################################
################################################################################
# t_0 : time origin
0
# T_max : maximal maturity (in year)
2
# y_min : --> S_min=exp(y_min)     
-7
# y_max : --> S_max=exp(y_max)
7
# N : number of space steps of the fine grid
400
# M : number of time steps of the fine grid
100
# gridType : type of the y_fineGrid (0 for regular, 1 for tanh)
0
# theta : type of scheme used
.5
################################################################################
################################## OPTIMIZATION ################################
################################################################################
# choice_optim : choice of the optimization method (1-->QN, 2-->QN with bounds)
2
################################################################################
################################# FILES IN/OUT #################################
################################################################################
# name_in_optim : name of the file containing the optimization parameters
optim2.in
# name_in_data : name of the file containing the data prices
sp500prices.data
# name_in_sigmainit_ddl : name of the file containing the degrees of freedom of
#   the initial sigma
sigmainit_n1m1_0_013.ddl
# name_out_sigmaest_ddl : name of the file containing the degrees of freedom of
#   the estimated sigma
v2_008_1_sigmaest_n1m1_0_013.ddl
# name_in_sigma_visu : name of the file containing the visualization parameters
#   Type "_" if no visualization is required
visusigma
# name_out_sigmainit_visu : name of the out file containing the initial sigma 
#   to be visualized (Type "_" is no visualization is required)
sigmainit_n1m1_0_013.visu
# name_out_sigmaest_visu : name of the out file containing the estimated sigma
#   to be visualized (Type "_" is no visualization is required)
v2_008_1_sigmaest_n1m1_0_013.visu
\end{verbatim}

The files optim2.in, sp500prices.data, sigmainit\_n1m1\_0\_013.ddl and 
visusigma are contained in the subdirectory ./src/examples. Here, we 
employ a Quasi-Newton algorithm with bounds 0.08-1 as we can see 
in the file optim2.in:

\begin{verbatim}
# pgtol : tolerance on the infinity norm of the projected gradient
0.00001
# factr : tolerance on the change of the objective function
10000000
# iprint : level of printed information
1
# maxCounter : maximum number of iterations
30
# sigma_min : minimal bound on sigma
0.08
# sigma_max : maximal bound on sigma
1.0
# lambda : coeff of F1 in the obj function : F = G+lambda*F1
0
\end{verbatim}

We also give an example of file optim.in which could be used 
with a Quasi-Newton algorithm without constraints, namely 
if ${\rm choice\_optim} = 1$:

\begin{verbatim}
# gradtol : tolerance on the relative gradient
0.0000000001
# steptol : tolerance on the relative change of x
0.0000000001
# verbosity : level of printed information (0 --> 3)
3
# saveSuccessiveXinFile : save succesive x0 in the file data.out (0 ou 1)
0
# maxCounter : maximum number of iterations
30
# lambda : coeff of F1 in the obj function : F = G+lambda*F1
0
\end{verbatim}

\subsection{rafsig.in}

This file rafsig.in could be used at the end of the previous calib 
execution in order to discretize the estimated volatility from 
a grid $1 \times 1$ to a grid $3 \times 3$ (multiscale 
approach).

\begin{verbatim}
################################################################################
################################# FILES IN/OUT #################################
################################################################################
# name_in_sigma1_ddl : name of the file containing the ddl
v2_008_1_sigmaest_n1m1_0_013.ddl
# nbsplit_y : number (integer) of subdivisions in y
3
# nbsplit_T : number (integer) of subdivisions in T
3
# name_out_sigma2_ddl : name of the out file containing the new ddl
v2_008_1_sigmainit_n3m3_0_013.ddl
\end{verbatim}

\subsection{visusig.in}

This is an example of file visusig.in allowing us to visualize the 
estimated volatility on a grid defined by visusigma.

\begin{verbatim}
################################################################################
################################# FILES IN/OUT #################################
################################################################################
# name_in_sigma_ddl : name of the file containing the ddl
v2_008_1_sigmaest_n1m1_0_013.ddl
# name_in_sigma_visu : name of the file containing the visualization parameters
visusigma
# name_out_sigma_visu : name of the out file containing sigma 
v2_008_1_sigmaest_n1m1_0_013.visu

\end{verbatim}

\subsection{impsig.in}

This is a file impsig.in used to compute the implied volatilities from 
the options prices defined in sp500prices.data.

\begin{verbatim}
################################################################################
#################### VARIABLES S_0, R, Q, OPTIONTYPE AND T_0 ###################
################################################################################
# S_0 : price of the underlying asset at t=t_0
590
# r : risk-free rate     
0.06
# q : dividends (continuously compounded) 
0.0262
# optionType : type of the option (1 for call, 0 for put)  
1
# t_0 : time origin
0
################################################################################
################################# FILES IN/OUT #################################
################################################################################
# name_in_data : name of the file containing Ki, Tj and V_ij
sp500prices.data
# name_out_sigma : name of the file containing Ki, Tj and impsigma_ij
#v2_008_1_sigmaest_n12m12_0_013.t0175impsigdata
sp500impsig2.data
\end{verbatim}

\section{Bugs}

Be careful of the following bug. After execution of rafsig, 
the numbers $y_0$ and $y_n$ written in the file .ddl are 
truncated by the fprintf function and they do not correspond 
exactly to $y_{min}$ and $y_{max}$ given in calib.in. For 
example, -5.298317 is written instead of -5.29831736654804. 
This fact produces a bug if you want to execute calib. To 
solve this problem, you have to edit the file .ddl and 
write manually $y_0$ and $y_n$ (5.29831736654804 and 
-5.29831736654804 in our example).

Be careful to enter numbers and names of files in the 
files simul.in, calib.in, etc. without any blank or 
tab at the end, because our syntaxic analyzer does not 
take into account these cases.

\newpage
\bibliography{biblio}
\bibliographystyle{plain}

\end{document}
