\section{Introduction}

Le mod\`ele de Black-Scholes suppose que l'\'evolution du prix de 
l'actif est r\'egi par l'\'equation diff\'erentielle stochastique 
suivante (cf. \cite{bla:jpe:73})~:
\begin{eqnarray*}
dS_t = S_t (\mu dt + \sigma dB_t),
\end{eqnarray*}
o\`u la volatilit\'e $\sigma$ du sous-jacent est suppos\'ee 
constante et o\`u $B_t$ est un mouvement brownien. A partir de ce 
mod\`ele de diffusion de l'actif, on peut d\'eterminer le prix des 
options sur cet actif. Inversement, la volatilit\'e $\sigma$ \'etant 
le seul param\`etre non observable de ce mod\`ele, on s'int\'eresse 
au calcul de $\sigma$ \`a partir du prix d'une option $(K,T)$ 
observ\'e sur le march\'e. La volatilit\'e ainsi obtenue est 
appel\'ee volatilit\'e implicite (not\'ee $\sigma_{BS}(K,T)$). 

Lorsque l'on confronte ce mod\`ele aux donn\'ees r\'eelles, on 
r\'ealise ses limites~: la volatilit\'e implicite $\sigma_{BS}(K,T)$ 
est tr\`es sensible aux variations du strike $K$ et de la maturit\'e 
$T$. Or, la volatilit\'e d'un actif ne peut pas d\'ependre de 
l'option que l'on consid\`ere. Ce ph\'enom\`ene est connu sous le 
nom de ``smile'' (cf.~\cite{dup:risk:94}). Il est important d'estimer 
la volatilit\'e le plus justement possible pour ensuite pouvoir 
\'evaluer des options plus complexes. Il faut donc trouver une autre 
repr\'esentation de $\sigma$. 

De nombreux articles se sont int\'eress\'es \`a ce probl\`eme 
(\cite{lag:jcf:97}, \cite{col:jcf:99}, \cite{jac:jcf:98}, 
\cite{ave:amf:97}), en proposant un nouveau processus de diffusion 
pour le sous-jacent (cf. par exemple \cite{dup:risk:94}, 
\cite{lag:jcf:97})~: 
\begin{eqnarray*}
dS_t = S_t (\mu dt + \sigma(S_t,t) dB_t),
\end{eqnarray*}
o\`u la volatilit\'e $\sigma(S_t,t)$ d\'epend maintenant du cours 
du sous-jacent et du temps. Dans le mod\`ele de base de Black-Scholes 
(avec $\sigma$ constant), on peut facilement d\'eduire $\sigma_{BS}$ 
du prix de l'option. Dans le cas de la volatilit\'e locale, 
d\'eterminer la nappe $\sigma(S_t,t)$ est beaucoup plus d\'elicat. 
Cela revient en fait \`a r\'esoudre un probl\`eme inverse 
sous-d\'etermin\'e et mal pos\'e. 

Dans la m\'ethode pr\'esent\'ee ici, la fonction de volatilit\'e 
appara\^{\i}t comme coefficient d'une EDP (EDP de Dupire) 
v\'erifi\'ee par le prix de l'option. On cherche \`a d\'eterminer 
ce coefficient \`a partir de l'information dont on dispose concernant 
le prix gouvern\'e par l'EDP. En principe, ce probl\`eme inverse 
peut \^etre r\'esolu si l'on a assez d'information sur les prix 
d'options, mais dans la pratique~: 
\begin{itemize}
\item le probl\`eme inverse est sous-d\'etermin\'e car le march\'e 
ne fournit qu'une quantit\'e limit\'ee d'information,
\item le probl\`eme est mal pos\'e car la volatilit\'e ne 
d\'ependra pas contin\^ument des donn\'ees du march\'e.
\end{itemize}
Pour rem\'edier \`a ces obstacles, nous utilisons une technique de 
r\'egularisation qui permet de faire un compromis entre la 
pr\'ecision (les donn\'ees simul\'ees avec notre mod\`ele sont 
proches des donn\'ees du march\'e) et la stabilit\'e en recherchant 
une volatilit\'e suffisamment r\'eguli\`ere. Dans un premier temps, 
nous ajoutons un terme de r\'egularisation 
$\lambda \| \sigma \|^2$ \`a la fonction co\^ut mesurant l'\'ecart 
entre les donn\'ees simul\'ees et les donn\'ees du march\'e. La 
difficult\'e revient \`a d\'eterminer le coefficient $\lambda$ le 
mieux adapt\'e. Apr\`es avoir montr\'e des premiers tests de 
calibration sur donn\'ees synth\'etiques, nous proposons une 
autre m\'ethode pour r\'egulariser le probl\`eme~: l'approche 
multi\'echelle qui est d\'ecrite \`a la 
section~\ref{SSEC:MULTIECHELLE}.

Plusieurs articles ont d\'ej\`a attaqu\'e ce probl\`eme avec le 
m\^eme type d'approche ``probl\`eme inverse'' (\cite{lag:jcf:97}, 
\cite{jac:jcf:98}, \cite{col:jcf:99}). La m\'ethode propos\'ee ici 
diff\`ere cependant en plusieurs points, dont nous discuterons 
l'opportunit\'e~: 
\begin{itemize}
\item la formulation du probl\`eme direct (EDP de Dupire et non de 
Black-Scholes),
\item le param\'etrage de $\sigma$ (spline bicubique non contraint),
\item la m\'ethode d'optimisation (Quasi-Newton avec calcul exact 
du gradient, aux erreurs d'arrondis pr\`es).
\end{itemize}

Dans un premier temps, nous pr\'esentons une m\'ethode de 
r\'esolution du probl\`eme direct de ``pricing'' (EDP de Dupire). 
S'ensuit une discussion concernant la r\'egularisation du probl\`eme 
et le choix de la m\'ethode d'optimisation \`a utiliser. Notre 
choix se porte sur une approche multi\'echelle et une m\'ethode de 
Quasi-Newton avec bornes, coupl\'ees avec un param\'etrage de la 
volatilit\'e par des splines bicubiques. Nous validons notre 
m\'ethode de calibration sur deux jeux de donn\'ees synth\'etiques 
et sur un jeu de donn\'ees r\'eelles observ\'ees sur le march\'e.
