\documentclass[a4paper,12pt]{article}
\usepackage{amssymb,amsmath}
\begin{document}
\begin{center}
{\Large Calibration by a weighted Monte Carlo method .}

\


\

\Large Yael BEER-GABEL

\Large June 2003
\end{center}
\

\

This calibration method introduced in \cite{ABFGKN} consists in generating
$\nu$ sample paths according to some a priori model and in assigning non
uniform weights to these paths according to the market prices of some
Put options.



\section*{The a priori model.}

\

The sample paths are generated according to the following stochastic
volatility model:
\begin{equation*}
\begin{cases}
dS_t=S_t((r-q) dt+\sigma_t dB^1_t)\\
d\sigma_t=-\beta (\sigma_t - \sigma_0) dt + \alpha dB^2_t \\
d<B^1,B^2>_t =\rho dt, -1\leq \rho \leq 1
\end{cases}
\end{equation*}
where
\begin{itemize}
\item{r is the risk free interest rate.}
\item{q is the dividend rate}
\item{$\sigma_t$ is the stochastic volatility.}
\item{$B^1_t,B^2_t$ are  brownian motions.}
\item{$\rho$ is the correlation between the brownian motions.}
\item{$\sigma_0$ is the mean volatility.}
\item{$\alpha$ is the perturbation .}
\item{$\beta>0$ is the rate of return to mean.}
\end{itemize}
More precisely,we generate discretized sample-paths by the standard
Euler scheme with timestep $\Delta t$:
\begin{equation}
\begin{cases}
\overline{S}_{(n+1)\Delta t} =  \overline{S}_{n\Delta
t}(1+(r-q)\Delta t +\overline{\sigma}_{n\Delta
  t}\sqrt{\Delta t}g^1_{n+1}) , n=1,\hdots,M\\
\overline{\sigma}_{(n+1)\Delta t}= \sigma_0+
(\overline{\sigma}_{n\Delta t}-\sigma_0)(1- \beta \Delta t)+
\alpha \sqrt{\Delta t}(\rho g^1_{n+1}+\sqrt{1-\rho^2} g^2_{n+1})
\end{cases}
\end{equation}
where $(g^1_{n},g^2_{n})_{1\leq n \leq M}$ are i.i.d according to
$\mathcal{N}_{2}(0,Id)$.

\

\

\begin{flushleft}
We construct a set of $\nu$ sample paths of (1) which we denote
by
$$\omega^{(i)}=(\overline{S}_{\Delta t}(\omega^{(i)}),\hdots,\overline{S}_{M \Delta t}(\omega^{(i)})) , i=1,\hdots,\nu$$
\end{flushleft}
\section*{Computation of the weights.}
The choice of the weights $(p_i)_{1\leq i \leq \nu}$ is done by
minimizing the Kullback-Leibler relative entropy
$D(p\vert{u})=\ln(\nu)+\sum_{i=1}^{\nu}{p_i\ln(p_i)}$ with respect to the
uniform distribution under the constraint that the corresponding price
of the benchmark puts is equal to their market prices.\\
More precisely, let $P_1,...,P_N$ denote the market prices of N
european puts,  and $g_{1j},g_{2j},\hdots,g_{\nu j} ,
j=1,\hdots,N$ represent the present values of the cashflows of the
\emph{j}th benchmark along the different paths, i.e
$g_{ij}=e^{-rk_{j}\Delta t}(\overline{S}_{k_{j}\Delta
  t}(\omega^{(i)})-K_{j})^{+}$, where $K_{j}$ is the strike of the \emph{j}th
benchmark and $k_{j}\Delta t$ is its maturity.
\\
The price relations for the benchmark instruments can be written in the
form
\begin{equation}
\sum_{i=1}^{\nu}{p_i g_{ij}}=P_j, j=1,\hdots,N
\end{equation}
The mathematical problem is to minimize the convex function of p
\begin{equation*}
D(p\vert{u})=\ln(\nu)+\sum_{i=1}^{\nu}{p_i\ln(p_i)}
\end{equation*}
under the linear constraints (2).
\\
Let's introduce Langrange multipliers $(\lambda_1,...,\lambda_N)$. The
problem is equivalent to find p which maximizes $$ \min_{\lambda \in
  \mathbb{R}^N} {\bigg(}-D(p\vert{u})+\sum_{j=1}^{N}{\lambda_j{\bigg(}\sum_{i=1}^{\nu}{p_ig_{ij}}-P_j}{\bigg)\bigg )}$$ and can be solved by considering the dual formulation
\begin{equation*}
\min_{\lambda}\bigg(\max_{p}\bigg(-D(p\vert{u})+\sum_{j=1}^{N}{\lambda_j\bigg(\sum_{i=1}^{\nu}{p_ig_{ij}}-P_j}\bigg)\bigg)\bigg
)
\end{equation*}
A straightforward argument shows that the probability vector that
realizes the supremum for each $\lambda$ has the Boltzmann-Gibbs form
\begin{equation*}
p_i=p(\omega^{(i)})=\frac{1}{Z(\lambda)}e^{\sum_{j=1}^{N}{g_{ij}\lambda_j}}
\end{equation*}
where
\begin{equation*}
Z(\lambda)=\sum_{i=1}^{\nu}{e^{\sum_{j=1}^{N}{g_{ij}\lambda_j}}}
\end{equation*}
Then the optimal $\lambda$ minimizes
\begin{equation*}
W(\lambda)=\ln
{\bigg(}\sum_{i=1}^{\nu}{e^{\sum_{j=1}^{N}{g_{ij}\lambda_j}}}{\bigg)}-\sum_{j=1}^{N}{\lambda_jP_j}
\end{equation*}
It is also interesting not to satisfy exactly the constraint by
choosing $\lambda$ which minimizes the penalized
function:$$\ln\bigg(\sum_{i=1}^{\nu}{e^{\sum_{j=1}^{N}{g_{ij}\lambda_j}}}\bigg)-\sum_{j=1}^{N}{\lambda_j
  P_j}+\frac{1}{2}\sum_{j=1}^{N}{w_j\lambda_j^2}$$ \\
We recover the previous case by choosing $w_j= 0 , 1 \leq j
\leq N$.
\section*{Numerical resolution}
We first renormalize the payoff by introducing $\alpha_j=\sup_{1 \leq i
  \leq \nu}|g_{ij}-P_j|$ and setting
  $\tilde{g}_{ij}=\frac{g_{ij}-P_{j}}{\alpha_j}$.\\
We find $(\tilde{\lambda}_1,...,\tilde{\lambda}_N)$ which
minimizes
  $$\ln\bigg(\sum_{i=1}^{\nu}{e^{\sum_{j=1}^{N}{\tilde{\lambda}_j\tilde{g}_{ij}}}}\bigg)+\frac{1}{2}\sum_{j=1}^{N}{
  \frac{w_j}{\alpha_j^2}\tilde{\lambda}_j^2}$$ by a conjugate gradient algorithm
  and deduce the corresponding weights $$ p_i=\frac{1}{\tilde{Z}(\tilde{\lambda})}e^{\sum_{j=1}^{N}{\tilde{g}_{ij}\tilde{\lambda}_j}}$$
where
\begin{equation*}
\tilde{Z}(\tilde{\lambda})=\sum_{i=1}^{\nu}{e^{\sum_{j=1}^{N}{\tilde{g}_{ij}\tilde{\lambda}_j}}}
\end{equation*}

\begin{thebibliography}{1}
\bibitem{ABFGKN}
M.\textsc{Avellaneda}, R.\textsc{Buff}, C.\textsc{Friedman},
N.\textsc{Grandchamp}, L.\textsc{Kruk}, J.\textsc{Newman}.
\emph{Weighted Monte Carlo : A New Technique for Calibrating
 Asset-Pricing Models}.
International Journal of Theoretical and Applied Finance
Vol. 4, No. 1 (2001) 91-119
\end{thebibliography}

\end{document}
