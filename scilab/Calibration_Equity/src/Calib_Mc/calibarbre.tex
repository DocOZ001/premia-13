\documentclass[11pt]{article}
\usepackage{amsmath,amssymb}
\begin{document}
\subsection*{Calibration of a trinomial recombining tree}
The aim is to calibrate a trinomial recombining tree with $N$ time
steps. At time $n\in\{0,1,\hdots,N\}$, the possible values of the stock $S_n$ are
$x_k=S_0\Delta^k$ for $-n\leq k\leq n$ where $\Delta>1$ denotes the
constant logarithmitic step. One has to compute for
$n\in\{0,1,\hdots,N-1\}$ the $3(2n+1)$ transition
probabilities $$\left(\begin{array}{ccc}p_n(k,k-1)&=&{\mathbb
  P}(S_{n+1}=x_{k-1}|S_n=x_k)\\p_n(k,k)&=&{\mathbb
  P}(S_{n+1}=x_{k}|S_n=x_k)\\ p_n(k,k+1)&=&{\mathbb
  P}(S_{n+1}=x_{k+1}|S_n=x_k)\end{array}\right)_{-n\leq k\leq n}.$$
consistent with the market prices of some
Call options. The interest rate $r$ is assumed to be
constant and such that $\frac{1}{\Delta}-1<r<\Delta -1$. Of course for $-n\leq k\leq n$, one has
$$p_n(k,k-1)+p_n(k,k)+ p_n(k,k+1)=1$$
which amounts to $2n+1$ equations. The
necessity for the actualized stock price to be a martingale gives $2n+1$
more equations : for $-n\leq k\leq n$,
$$p_n(k,k-1)/\Delta+p_n(k,k)+p_n(k,k+1)\Delta=1+r.$$
One can then express $p_n(k,k)$ and $p_n(k,k-1)$ in terms of
$p_n(k,k+1)$ :
\begin{equation}
\begin{cases}p_n(k,k)=1+\frac{\Delta r}{\Delta -1}-(\Delta +1)p_n(k,k+1)\\
      p_n(k,k-1)=\Delta\left(p_n(k,k+1)-\frac{r}{\Delta -1}\right)
   \end{cases}
\label{eq1}\end{equation}
The non-negativity of $p_n(k,k+1)$ and the above expressions of
$p_n(k,k)$ and $p_n(k,k-1)$ is equivalent to \begin{equation}
  p_n(k,k+1)\in\left[\frac{r}{\Delta -1},\frac{\Delta(1+r)-1}{\Delta^2-1}\right]\label{condprob}.\end{equation}
The lacking equations come from the knowledge of the market prices
$C(n+1,k)$ of the $2n+1$ European Call options with maturity $n+1$ and
Strikes $x_k$ for $k\in\{-n,-n+1,\hdots,n-1,n\}$. Indeed, equations 
\begin{equation}
   (1+r)^{n+1}C(n+1,k)=\sum_{l=k+1}x_k(\Delta^{l-k}-1){\mathbb P}^*(S_{n+1}=x_l),\;-n\leq
k\leq n\label{prixcall}
\end{equation} form a triangular system of equations in the variables
${\mathbb P}^*(S_{n+1}=x_l),\;-n+1\leq l\leq n+1$ which is easily solved. In
addition, the convexity of the prices of the Call options with maturity
$n+1$ with respect to the Strike which holds for arbitrage reasons implies
$$C(n+1,k)\leq
\frac{1}{\Delta+1}C(n+1,k+1)+\frac{\Delta}{\Delta+1}C(n+1,k-1)$$
 since
 $x_k=\frac{1}{1+\Delta}x_{k+1}+\frac{\Delta}{1+\Delta}x_{k-1}$. Using
 \eqref{prixcall}, one
 obtains that the inequality writes 
$\frac{\Delta}{1+\Delta}x_{k-1}(\Delta -1){\mathbb P}^*(S_{n+1}=x_{k})\geq 0$.
Hence the solution
${\mathbb P}^*(S_{n+1}=x_l),\;-n+1\leq l\leq n+1$ of the triangular
system consists in non-negative terms for arbitrage reasons.\par
Now
supposing that the calibration procedure proceeds inductively on $n$,
the risk-neutral probabilities ${\mathbb P}^*(S_{n}=x_k)$ are known from
the previous steps and one can write $2n+1$ forward equations linking
the risk neutral law of $S_{n+1}$ to the one of $S_n$ : for $-n+1\leq k\leq n+1$,
\begin{align}{\mathbb
    P}^*(S_{n+1}=x_k)&={\mathbb P}^*(S_n=x_{k-1})p_n(k-1,k)\notag\\&+1_{\{k\leq n\}}{\mathbb P}^*(S_n=x_{k})p_n(k,k)\notag\\&+1_{\{k\leq n-1\}}{\mathbb P}^*(S_n=x_{k+1})p_n(k+1,k).
\label{eq3}\end{align}
Now from (\ref{eq3}) for $k=n+1$, one obtains $p_n(n,n+1)$. From
(\ref{eq1}) one deduces $p_n(n,n)$ and $p_n(n,n-1)$. Next from
(\ref{eq3}) for $k=n$, one obtains $p_n(n-1,n)$ and one deduces
$p_n(n-1,n-1)$ and $p_n(n-1,n-2)$ from (\ref{eq1}) and so on. Hence the
transition probabilities at time $n$  are easily computed by backward
induction on $k$ from $n+1$ to $-n+1$.\par
Apart from the practical difficulty of interpolation between the strikes
and maturities traded on the market to obtain the market prices $C(n+1,k)$ for
$0\leq n\leq N-1$ and $-n\leq k\leq n$, nothing garanties from a
theoretical point of view that the values $p_n(k,k+1)$ obtained by the
previous procedure satisfy \eqref{condprob} which is
equivalent to the non-negativity of $p_n(k,k+1)$ and of $p_n(k,k)$ and
$p_n(k,k-1)$ given by \eqref{eq1}. Hence some numerical procedure, like
projection on the interval, has to ensure this condition. Then the
calibrated probabilities of $\{S_{n+1}=k\}$ for $-n+1\leq k\leq n+1$ are no longer equal to the ones
derived from the Call option prices. And at each step of the induction
on $n$, after computation of the transition probabilities
$p_n(k,k+1),\;p_n(k,k),\;p_n(k,k-1)$ for $-n\leq k\leq n$ the tree values of
${\mathbb P}^*(S_{n+1}=k)$ for $-(n+1)\leq k\leq n+1$ used in the next
step have to be computed from \eqref{eq3} (note that the computation of
${\mathbb P}^*(S_{n+1}=-n)$ and ${\mathbb P}^*(S_{n+1}=-(n+1))$ from
\eqref{eq3} is always necessary since these probabilities are not
deduced from the Call options prices). 
\begin{thebibliography}{1}
\bibitem{Dupire}
B.~Dupire,
\emph{Pricing with a Smile}.
Risk Magazine 7, 18-20, 1994
\end{thebibliography}
\end{document}
