
\section{Libor market model: a stochastic volatility extension}


This section presents a stochastic volatility extension of the libor market model, we recall the main equations of the article within the notation of this document

\subsection{The model}

Under the risk neutral measure $Q$ the zero coupon bond follows the dynamic

\ban
\frac{dB(t,T)}{B(t,T)}&=&r(t)dt+\sqrt{V_t}\sigma_B(t,T)'dW_t\\
dV_t&=&\kappa(\theta - V_t)dt+\epsilon \sqrt{V_t}dZ_t
\ean

where $\left( W_t;t\geq 0 \right)$ is a $d$ dimensional brownian motion under $Q$, $\left( Z_t;t\geq 0 \right)$ is a $1$ dimensional brownian motion under $Q$, and $\sigma_B(t,T)$ is a $1*d$ vector. If we choose $\sigma_B(t,T)$ deterministic then we get the model proposed by Collin-Dufresne and Goldstein \cite{collinDufresneGoldstein}. 

we have 

\ban
\frac{dL(t,T_j,\tau)}{L(t,T_j,\tau)}&=&\sqrt{V_t}\gamma(t,T_j,\tau)'[dW^Q_t-\sqrt{V_t}\sigma_B(t,T_{j+1})dt] \\
dV_t&=&\kappa(\theta - V_t)dt+\epsilon \sqrt{V_t}dZ_t
\ean

with

\ba
\gamma(t,T_j,\tau)=\frac{1+\tau L(t,T_j,\tau)}{\tau L(t,T_j,\tau)}[\sigma_B(t,T_j)-\sigma_B(t,T_{j+1})] \label{forwardvolbondvol}
\ea


In the libor market model we make the hypothesis that 

\begin{center}
\fbox{ $\lbrace \gamma(t,T_j,\tau) ; \; t \geq 0 ;\; j=1..M \rbrace $ are deterministic functions. }
\end{center}


We note $\gamma(t,T_j,\tau)=(\gamma^1(t,T_j,\tau),\gamma^2(t,T_j,\tau),..,\gamma^d(t,T_j,\tau) )$ \\

From (\ref{forwardvolbondvol}) and under the hypothesis $\sigma_B(t,T_1)=0$ we obtain

\ban
\sigma_B(t,T_{j+1})=-\sum_{k=1}^j\frac{1+\tau L(t,T_j,\tau)}{\tau L(t,T_j,\tau)}\gamma(t,T_k,\tau) 
\ean

the correlation between the forward rate factors and the volatility factor is given by

\ban
\frac{\gamma(t,T_j,\tau)'dW_t}{||\gamma(t,T_j,\tau)||}dZ_t=\rho_j(t)dt
\ean

If we note $W_t=(W^1_t,..,W^d_t)$ and $dW^i_tdZ_t=\rho^idt$ we have 

\ba
||\gamma(t,T_j,\tau)||\rho_j(t)dt&=& \gamma(t,T_j,\tau)'dW_tdZ_t\\
&=&\sum_{i=1}^d \rho^i\gamma^i(t,T_j,\tau) dt \label{rhovol}
\ea



if we note $Q^{T_{j+1}}$ the probability measure associated with $B(t,T_{j+1})$  as numeraire then we have

\[
\left\lbrace 
\begin{array}{l}
\frac{dL(t,T_j,\tau)}{L(t,T_j,\tau)}=\sqrt{V_t}\gamma(t,T_j,\tau)'dW^{Q^{T_{j+1}}}_t \\
dV_t=\kappa(\theta - (1+\frac{\epsilon}{\kappa}\xi_j(t))V_t)dt+\epsilon \sqrt{V_t}dZ^{Q^{T_{j+1}}}_t
\end{array}
\right.
\]

where $W^{Q^{T_{j+1}}}_t$ resp. $Z^{Q^{T_{j+1}}}_t$ is a $1*d$ resp. $1$ dimensional brownian motion under $Q^{T_{j+1}}$ and 

\ban
\xi_j(t)=\sum_{k=1}^j\frac{\tau L(t,T_k,\tau)}{1+\tau L(t,T_k,\tau)}\rho_k(t)||\gamma(t,T_k,\tau)|| 
\ean

the authors propose to freeze this stochastic process and  define 

\ban
\xi_j^0(t)=\sum_{k=1}^j\frac{\tau L(0,T_k,\tau)}{1+\tau L(0,T_k,\tau)}\rho_k(t)||\gamma(t,T_k,\tau)|| 
\ean

\ban
\tilde \xi_j(t)&=&1+\frac{\epsilon}{\kappa} \xi_j(t)\\
\tilde \xi_j^0(t)&=&1+\frac{\epsilon}{\kappa}\xi_j^0(t)
\ean

thus the dynamic is given by

\[
\left\lbrace 
\begin{array}{l}
\frac{dL(t,T_j,\tau)}{L(t,T_j,\tau)}=\sqrt{V_t}\gamma(t,T_j,\tau)'dW^{Q^{T_{j+1}}}_t \\
dV_t=\kappa(\theta - \tilde \xi_j^0(t)V_t)dt+\epsilon \sqrt{V_t}dZ^{Q^{T_{j+1}}}_t
\end{array}
\right.
\]

\subsubsection{ Moment generating function for the caplet}

Computing the moment generating function for $X_u=ln \frac{L(u,T_j,\tau)}{L(t,T_j,\tau)}$ , define 

\ban
\phi(t,X_t,V_t,z)=E^{Q^{T_{j+1}}}\left[ e^{zX_{T_j}}|{\cal F}_t \right]
\ean

The function $\phi(t,x,V,z)$ satisfies the pde

\[
\left\lbrace 
\begin{array}{l}
\partial_t\phi+\kappa(\theta - \tilde \xi_j^0(t)V)\partial_V\phi -\frac{1}{2}||\gamma(t,T_j,\tau)||^2V\partial_x\phi \\
+\frac{1}{2}\epsilon^2V\partial^2_{VV}\phi+\epsilon\rho_j(t)V||\gamma(t,T_j,\tau)||\partial^2_{Vx}\phi+\frac{1}{2}||\gamma(t,T_j,\tau)||^2V\partial^2_{xx}\phi=0\\
\phi(T,x,V,z)=e^{zx}
\end{array}
\right.
\]



we define the function 

\ba
\phi_T(z)=\phi(t,0,V_t,z) \label{fi0}
\ea


\subsubsection{ Moment generating function for the swaption}

For the swaption pricing: recall 

\ban
S(t,T_s,T_M)&=&\frac{B(t,T_s)-B(t,T_M)}{\sum_{j=s+1}^{M}\tau B(t,T_j)}\\
&=&\frac{1 - \prod_{j=s}^{M-1}\frac{1}{1+\tau L(t,T_j,\tau)} }{\sum_{j=s+1}^{M}\tau \prod_{k=0}^{j-1}\frac{1}{1+\tau L(t,T_k,\tau)}}
\ean

using Ito's lemma we deduce that 

\ban
dS(t,T_s,T_M)&=&\sum_{j=s}^{M-1}\frac{\partial S(t,T_s,T_M) }{\partial L(t,T_j,\tau)}L(t,T_j,\tau)\sqrt{V_t}\gamma(t,T_j,\tau)'[dW_t-\sqrt{V_t}\sigma_S(t)dt]\\
dV_t&=&\kappa(\theta - \tilde \xi_S(t)V_t)dt+\epsilon \sqrt{V_t}[dZ_t+\xi_S(t)dt]
\ean

with

\ban
\sigma_S(t)&=&\sum_{j=s}^{M-1}\alpha_j(t)\sigma_B(t,T_{j+1})\\
\tilde \xi_S(t)&=&1+\frac{\epsilon}{\kappa}\sum_{j=s}^{M-1}\alpha_j(t)\xi_j(t)\\
\alpha_j(t)&=&\frac{\tau B(t,T_{j+1})}{ \sum_{j=s}^{M-1} \tau B(t,T_{j+1})}\\
\frac{\partial S(t,T_s,T_M) }{\partial L(t,T_j,\tau)}&=&\frac{\tau S(t,T_s,T_M)}{(1+\tau L(t,T_j,\tau))}\left( \frac{B(t,T_M)}{B(t,T_s)-B(t,T_M)}+\frac{\sum_{k=j+1}^M \tau B(t,T_k)}{\sum_{j=s+1}^{M}\tau B(t,T_j)}  \right)
\ean

the dynamic of the forward swap rate is given by

\[
\left\lbrace 
\begin{array}{l}
dS(t,T_s,T_M)=\sum_{j=s}^{M-1}\frac{\partial S(t,T_s,T_M) }{\partial L(t,T_j,\tau)}L(t,T_j,\tau)\sqrt{V_t}\gamma(t,T_j,\tau)'dW^{Q^S}_t\\
dV_t=\kappa(\theta - \tilde \xi_S(t)V_t)dt+\epsilon \sqrt{V_t}dZ^{Q^S}_t
\end{array}
\right.
\]

with

\ban
dW^{Q^S}_t&=&dW_t-\sqrt{V_t}\sigma_S(t)dt\\
dZ^{Q^S}_t&=&dZ_t-\sqrt{V_t}\xi_S(t)dt
\ean

where $W^{Q^S}_t$ resp. $Z^{Q^S}_t$ is a $1*d$ dimensional resp $1$ dimensionnal brownian motion under $Q^S$.

 
freezing the volatility for the forward swap rate and the drift of the volatility we get


\[
\left\lbrace 
\begin{array}{l}
\frac{dS(t,T_s,T_M)}{S(t,T_s,T_M)}= \sum_{j=s}^{M-1}\omega_j(0)\sqrt{V_t}\gamma(t,T_j,\tau)'dW^{Q^S}_t\\
dV_t=\kappa(\theta - \tilde \xi_S^0(t)V_t)dt+\epsilon \sqrt{V_t}dZ^{Q^S}_t
\end{array}
\right.
\]

\ban
\omega_j(0)&=&\frac{\partial S(0,T_s,T_M) }{\partial L(0,T_j,\tau)}\frac{L(0,T_j,\tau)}{S(0,T_s,T_M)}\\
\tilde \xi_S^0(t)&=&1+\frac{\epsilon}{\kappa}\sum_{j=s}^{M-1}\alpha_j(0)\xi_j^0(t)
\ean

Computing the moment generating function for $X_u=ln\frac{S(u,T_s,T_M)}{S(t,T_s,T_M)}$ , define 

\ban
\phi(t,X_t,V_t,z)=E^{Q^S}\left[ e^{zX_T}|{\cal F}_t\right]
\ean

The function $\phi(t,x,V,z)$ satisfies the pde

\[
\left\lbrace 
\begin{array}{l}
\partial_t\phi+\kappa(\theta - \tilde \xi_S^0(t)V)\partial_V\phi -\frac{1}{2}||\gamma_{s,M}(t)||^2V\partial_x\phi \\
+\frac{1}{2}\epsilon^2V\partial^2_{VV}\phi+\epsilon\rho^S(t)V||\gamma_{s,M}(t)||\partial^2_{Vx}\phi+\frac{1}{2}||\gamma_{s,M}(t)||^2V\partial^2_{xx}\phi=0\\
\phi(T,x,V,z)=e^{zx}
\end{array}
\right.
\]

with

\ban
\gamma_{s,M}(t)&=&\sum_{j=s}^{M-1}\omega_j(0)\gamma(t,T_j,\tau)\\
\rho^S(t)&=&\frac{\sum_{j=s}^{M-1} \omega_j(0)||\gamma(t,T_j,\tau)||\rho_j(t) }{||\gamma_{s,M}(t)||}
\ean

furthermore the authors suggest, arguing a calibration objective not presented in the paper, to approximate

\ban
\rho^S(t)\sim\sum_{j=s}^{M-1}\omega_j(0)\rho_j(t)
\ean
 
In fact, this approximation is useless because only  $\rho^S(t)||\gamma_{s,M}(t)||$ is needed and (\ref{rhovol}) is used.


We define the function $\phi_T(z)$ by 

\ba
\phi_T(z)=\phi(t,0,V_t,z) \label{fi1}
\ea


\subsubsection{ Computing the moment generating function}

The pdes are identical as such we write both in a compact form

\[
\left\lbrace 
\begin{array}{l}
\partial_t\phi  +\kappa(\theta - \beta(t)V)\partial_V\phi -\frac{1}{2}\lambda(t)^2V\partial_x\phi\\
+\frac{1}{2}\epsilon^2V\partial^2_{VV}\phi+\epsilon\rho(t)V\lambda(t)\partial^2_{Vx}\phi+\frac{1}{2}\lambda(t)^2V\partial^2_{xx}\phi=0\\
\phi(T,x,V,z)=e^{zx} 
\end{array}
\right. 
\]


for the caplet

\ban
\beta(t)&=& \tilde \xi_j^0(t) \\
\lambda(t)&=&||\gamma(t,T_j,\tau)|| \\
\rho(t)&=&\rho_j(t)\\
\zeta(t)&=&||\gamma(t,T_j,\tau)||  \rho_j(t)
\ean


for the swaption

\ban
\beta(t) &=& \tilde \xi_S^0(t)\\
\lambda(t) &=& ||\gamma_{s,M}(t)|| \\
\rho(t) &=& \rho^S(t)\\
\zeta(t)&=&\rho^S(t)||\gamma_{s,M}(t)|| 
\ean

we emphazis the time dependence of the parameters. Looking for a solution of the form $\phi(t,x,V,z)=e^{A(t,z)+B(t,z)V+zx}$ we obtain the Riccati's equations

\ba
-\partial_tA(t,z)&=&\kappa \theta B(t,z)\\
-\partial_tB(t,z)&=&\frac{1}{2}\epsilon^2B(t,z)^2+(\rho(t)\epsilon\lambda(t)z-\kappa\beta(t) )B(t,z)+\frac{1}{2}\lambda(t)^2(z^2-z)\\ \label{riccati1}
&=&b_2(t)B(t,z)^2+b_1(t)B(t,z)+b_0(t)
\ea

with terminal conditions $A(T,z)=0$ and $B(T,z)=0$


Under the hypothesis that the volatility is piecewise constant and the maturity of the option is $T_N$ the solution of the above system is given by

\[
\left\lbrace 
\begin{array}{l}
B(t,z)=B(T_{i+1},z)+\frac{-b_1+d-2B(T_{i+1},z)b_2}{2b_2(1-ge^{d(T_{i+1}-t)})}(1-e^{d(T_{i+1}-t)})\\
A(t,z)=A(T_{i+1},z)+\frac{a_0}{2b_2}\left(  (-b_1+d)(T_{i+1}-t)-2ln\left(\frac{1-ge^{d(T_{i+1}-t)}}{1-g} \right)   \right)
\end{array}
\right. 
\]

for $t\in [T_i\; T_{i+1}]$ and $i \in \lbrace 0..N-1 \rbrace $ with 

\ban
A(T_N,z)&=&0\\
B(T_N,z)&=&0\\
a_0&=&\kappa \theta\\
b_1&=&\rho(T_i) \epsilon \lambda(T_i) z -\kappa \beta(T_i) \\
b_0&=&\frac{\lambda(T_i)^2}{2}(z^2-z)\\
b_2&=&\frac{\epsilon^2}{2}\\
d&=&\sqrt{\Delta}\\
\Delta&=&b_1^2-4b_0b_2\\
g&=&\frac{-b_1+d -2 B(T_{i+1},z)b_2}{-b_1-d -2 B(T_{i+1},z)b_2} 
\ean


{\bf Remark}: For computational prupose we embed the caplet/floorlet structure in the swpation structure. In fact we have


\ban
L(t,T_i,\tau)=S(t,T_i,T_{i+1})
\ean

as such for pricing a caplet or a swaption we will use the same algorithm.

\subsection{Derivatives pricing}

For the caplet $Cplt(t,T_M,K,\tau,N)$ we have

\ban
Cplt(t,T_M,K,\tau,N)&=&B(t,T_M+\tau)\tau N E^{Q^{T_M+\tau}}_t[(L(T_M,T_M,\tau)-K)_+]\\
&=&B(t,T_M+\tau)\tau N L(t,T_M,\tau) \left( I_1 - \frac{K}{L(t,T_M,\tau)}I_2 \right)
\ean

with

\ban
I_1&=&E^{Q^{T_M+\tau }}_t\left[ e^{ln\frac{L(T_M,T_M,\tau)}{L(t,T_M,\tau)}} {\bf 1}_{\lbrace \frac{L(T_M,T_M,\tau)}{L(t,T_M,\tau)}>\frac{K}{L(t,T_M,\tau)} \rbrace } \right]\\
I_2&=&E^{Q^{T_M+\tau }}_t\left[  {\bf 1}_{ \lbrace \frac{L(T_M,T_M,\tau)}{L(t,T_M,\tau)}>\frac{K}{L(t,T_M,\tau)} \rbrace }\right]
\ean



For the floorlet $Flt(t,T_M,K,\tau,N)$ we have

\ban
Flt(t,T_M,K,\tau,N)=B(t,T_M+\tau)\tau N L(t,T_M,\tau) \left( (1-I_2)\frac{K}{L(t,T_M,\tau)} -(1-I_1) \right)
\ean

For the european payer  swaption $Swpt(t,T_s,T_M,K,\tau,N)$ 

\ban
Swpt(t,T_s,T_M,K,\tau,N)= \sum_{i=s}^{M-1} B(t,T_{i+1}) \tau N S(t,T_s,T_M) \left( I_1 - \frac{K}{S(t,T_s,T_M)}I_2 \right)
\ean

\ban
I_1&=&E^{Q^S}_t\left[ e^{ln\frac{S(T_s,T_s,T_M)}{S(t,T_s,T_M)}} {\bf 1}_{\lbrace \frac{S(T_s,T_s,T_M)}{S(t,T_s,T_M)}>\frac{K}{S(t,T_s,T_M)} \rbrace } \right]\\
I_2&=&E^{Q^S}_t\left[  {\bf 1}_{ \lbrace \frac{S(T_s,T_s,T_M)}{S(t,T_s,T_M)}>\frac{K}{S(t,T_s,T_M)} \rbrace }\right]
\ean

For the  european receiver swaption $Swpt(t,T_s,T_M,K,\tau,N)$ 

\ban
Swpt(t,T_s,T_M,K,\tau,N)= \sum_{i=s}^{M-1} \tau B(t,T_{i+1}) N S(t,T_s,T_M) \left( \frac{K}{S(t,T_s,T_M)}(1-I_2) - (1-I_2) \right) 
\ean



{\bf Computing the integrals}


We have the following expressions for $I_1$ and $I_2$



\ban
I_1&=&\frac{1}{2}+\frac{1}{\pi}\int_0^{+\infty} \frac{ Im\lbrace e^{-iuln\left( \frac{K}{X(t)} \right)}   \phi_T(1+iu) \rbrace }{u}du\\
I_2&=&\frac{1}{2}+\frac{1}{\pi}\int_0^{+\infty} \frac{ Im\lbrace e^{-iuln\left( \frac{K}{X(t)} \right) }   \phi_T(iu) \rbrace }{u}du
\ean

where $\phi_T(u)$ is given by (\ref{fi0}) or (\ref{fi1}) depending on whether a swaption or a caplet is priced and $X(t)=L(t,T_M,\tau)$ resp. $X(t)=S(t,T_s,T_M)$ for the caplet/floorlet resp. the swaption (receiver or payer). It is also possible to compute the price using FFT method as in Carr, Madan\cite{carrMadan}, the computation time needed is approximatively twice faster.




\subsection{Numerical examples} 

For our numerical experiments we choose a two factors model with the following piecewise volatility structure: $\gamma(t,T_k,\tau)=(\gamma^1(t,T_k,\tau),\gamma^2(t,T_k,\tau))$.  


if $t\in [T_j \; T_{j+1}[$ 

\ban
\gamma^1(t,T_k,\tau)&=&0.2\\
\gamma^2(t,T_k,\tau)&=& \frac{0.01-0.05e^{-0.1(j-k)}}{\sqrt{0.04+0.00075j}}
\ean

and 

\ban
dW^1_tdZ_t&=&\rho^1dt=0.5 dt\\
dW^2_tdZ_t&=&\rho^2dt=0.2 dt
\ean

the yield curve is flat at $5\%$, $V_0=1$, $\epsilon=0.6$, $\kappa=1$ and $\theta=1$.
\begin{center}
{\bf Swaption payer prices in bps}\\
\begin{tabular}{|cccc|}\hline
{\small swaption maturity}  & Tenor     & strike      &   price   \\ \hline
  1      &  1    &    ATM   &  64.519   \\  
  1      &  5    &    ATM   &  405.221    \\
  1      &  10   &    ATM   &  1179.612    \\
  3      &  1    &    ATM   &  116.830   \\  
  3      &  5    &    ATM   &  739.835    \\
  3      &  10   &    ATM   &  2057.297     \\
  5      &  1    &    ATM   &   161.735  \\  
  5      &  5    &    ATM   &   1009.870 \\
  5      &  10   &    ATM   &   1904.210  \\ \hline
  1      &  1    &    0.8 ATM   &   114.683  \\  
  1      &  5    &    0.8 ATM   &    609.080 \\
  1      &  10   &    0.8 ATM   &    1472.062  \\
  3      &  1    &    0.8 ATM   &    151.380 \\  
  3      &  5    &    0.8 ATM   &    869.485 \\
  3      &  10   &    0.8 ATM   &    2201.807  \\
  5      &  1    &    0.8 ATM   &    185.766 \\  
  5      &  5    &    0.8 ATM   &    1087.164 \\
  5      &  10   &    0.8 ATM   &    2257.460  \\ \hline
  1      &  1    &    1.2 ATM   &    34.655 \\  
  1      &  5    &    1.2 ATM   &    267.585 \\
  1      &  10   &    1.2 ATM   &    954.980  \\
  3      &  1    &    1.2 ATM   &    91.083 \\  
  3      &  5    &    1.2 ATM   &    636.496 \\
  3      &  10   &    1.2 ATM   &    1934.698  \\
  5      &  1    &    1.2 ATM   &    142.306  \\  
  5      &  5    &    1.2 ATM   &    944.592 \\
  5      &  10   &    1.2 ATM   &    1623.445  \\ \hline
\end{tabular} 
\end{center} 


\subsection{Programming interface}


\subsubsection{C API of the pricer}

The function name is: 
\small{
\begin{verbatim}
double lmm_swaption_payer_stoVol_pricer(tenor ,numFac ,swaptionMat , swapMat , percent) 
\end{verbatim}
}

Arguments description:
\begin{itemize}
\item $tenor$ is the period in years of the rate (usually 3 or 6 months); type:double 
\item $numFac$ is the number of factors max 2; type: int
\item $swaptionMat$ is the swaption maturity in years; type: double
\item $swapMat$ is the swap maturity in years ; type: double 
\item $percent$ the strike will be equal to $(1+percent/100)*atm\_strike$ with atm\_strike the At The Money strike ; type: double 
\end{itemize}

{\bf Remarks:}
\begin{enumerate}
\item To price a caplet just call the function with $swapMat=swaptionMat+tenor$
\item $swapMat$ must be equal to $k*tenor$ with $k$ an integer
\item $swaptionMat$ must be equal to $k*tenor$ with $k$ an integer
\end{enumerate}

\subsubsection{calling the stochastic volatility pricer from a C program}

\small{

\begin{verbatim}
/*************************************************************************
 *
 *  example: - using the pricer function in a program
 *           - stochastic volatility model
 *
 *
 *************************************************************************/

#include"stdio.h"
#include"math.h"

#include"lmm_stochastic_volatility.h"

int main()
{

  float  tenor=0.5;         // period of the rate usually 3 months or 6 months
  int numFac=2;             // number of factors: dim of the brownian motion of the rates   
  double  swaptionMat=5.;   // swpation maturity
  double  swapMat=10.;      // swap maturity -> the tenor of swaption is  swapMat - swaptionMat
  double percent=-20.;      // the strike will be equal to (1+percent/100)*atm_strike
  double price;

  printf(" Payer swaption with maturity: %lf \n",swaptionMat );
  printf(" on a swap rate with maturity: %lf  (tenor equal to %lf)  \n", swapMat , 
	                                                   swapMat - swaptionMat);
  printf(" the strike is equal to (1+%lf) of the ATM strike \n", percent/100.); 
  printf(" the period of the underlying libor rate is %lf \n" , tenor );
  price=lmm_swaption_payer_stoVol_pricer(tenor ,numFac ,swaptionMat , swapMat , percent);
  printf(" the price in bps is : %lf \n", price*10000 ) ;

  
  return(1);

} 

\end{verbatim}

}


we obtain the following result:

\small{

\begin{verbatim}
$ lmm_stochastic_volatility_example 
 Payer swaption with maturity: 5.000000 
 on a swap rate with maturity: 10.000000  (tenor equal to 5.000000)  
 the strike is equal to (1+-0.200000) of the ATM strike 
 the period of the underlying libor rate is 0.500000 
 the price in bps is : 1087.164699 
$ 
\end{verbatim} 
}



\subsubsection{A Scilab function for the stochastic volatility pricer}

We defined a scilab function (in file lmm\_scilab.sci) for the stochastic volatility pricer which interface is as follow:


\small{
\begin{verbatim}
lmm_swpt_stovol_sci(period , nb_fac , swpt_mat , swp_mat , perct)
\end{verbatim} 
}

It returns the price in {\it bps} of the option and the input parameters are 

\begin{itemize}
\item {\it period} is the period length of the rate; type:double
\item {\it nb\_fac} is the number of factors; type: int
\item {\it swpt\_mat} is the swpation maturity in years; type: double
\item {\it swp\_mat} is th swap maturity in years; type: double
\item {\it perct} the strike will be equal to $(1+percent/100)*atm\_strike$ with atm\_strike the At The Money strike; type: double
\end{itemize}

{\bf Loading the scilab functions}: first you should compile the library, report to the README file. At the scilab ``File'' menu click on ``File Operations'' then select the file ``lmm\_scilab.sci'' and click on ``Getf'' buttom, it will produce something like this in ``scilex''
\small{
\begin{verbatim} 
-->;getf("/home/der_mif/jose/recherche/taux/prog/cprog/bgmPremia/code/lmm_scilab.sci");
\end{verbatim} 
}

all functions within this file are now available at the prompt or can be called from a scilab program. 
 
We illustrate the use of the function presented above. 

We obtained the following result from scilab-2.7:


\small{
\begin{verbatim}
-->b=lmm_swpt_stovol_sci(0.5, 2, 5., 10. , -20.0)         
shared archive loaded
Link done
 b  =
 
    1087.1647  
 
-->   
\end{verbatim} 
}


